% \setcourse{数学分析II}
% \setyear{2026\textasciitilde 2027}
% \setterm{春季}
\settotal{4}
% \setexamtype{期末考试}

\begin{document}

\makeexamtitle

% \begin{xiaosihao}


\section{%
  选择题:本题共 5 小题, 每小题 4 分, 共 20 分。
  在每小题给出的四个选项中, 只有一项是符合题目要求的。
}

% \noindent\scoringbox

\begin{question}
待写 \paren[]

\begin{choices}
\item 待写
\item 待写
\item 待写
\item 待写
\end{choices}
\end{question}

\begin{solution}
待写
\end{solution}

\begin{question}
待写 \paren[]

\begin{choices}
\item 待写
\item 待写
\item 待写
\item 待写
\end{choices}
\end{question}

\begin{solution}
待写
\end{solution}

\begin{question}
待写 \paren[]

\begin{choices}
\item 待写
\item 待写
\item 待写
\item 待写
\end{choices}
\end{question}

\begin{solution}
待写
\end{solution}

\begin{question}
待写 \paren[]

\begin{choices}
\item 待写
\item 待写
\item 待写
\item 待写
\end{choices}
\end{question}

\begin{solution}
待写
\end{solution}

\begin{question}
待写 \paren[]

\begin{choices}
\item 待写
\item 待写
\item 待写
\item 待写
\end{choices}
\end{question}

\begin{solution}
待写
\end{solution}


\section{填空题:本题共 5 小题, 每小题 4 分, 共 20 分。}

\examsetup{
  question/index = 1
}

% \noindent\scoringbox

\begin{question}
待写 \fillin[].
\end{question}

\begin{solution}
待写
\end{solution}

\begin{question}
待写 \fillin[].
\end{question}

\begin{solution}
待写
\end{solution}

\begin{question}
待写 \fillin[].
\end{question}

\begin{solution}
待写
\end{solution}

\begin{question}
待写 \fillin[].
\end{question}

\begin{solution}
待写
\end{solution}

\begin{question}
待写 \fillin[].
\end{question}

\begin{solution}
待写
\end{solution}


\section{计算题:本题共 2 小题, 共 12 分。本题应写出具体演算步骤。}

\examsetup{
  question/index = 1
}

\begin{question}[points = 6]
待写

% \noindent\scoringbox
\end{question}

\begin{solution}
待写
\end{solution}

\begin{question}[points = 6]
待写

% \noindent\scoringbox
\end{question}

\begin{solution}
待写
\end{solution}


\section{解答题:本题共 5 小题, 共 48 分。解答应写出文字说明或者证明过程。}

\examsetup{
  question/index = 1
}

\begin{question}[points = 6]
待写

% \noindent\scoringbox
\end{question}

\begin{solution}
待写
\end{solution}

\begin{question}[points = 10]
待写

% \noindent\scoringbox
\end{question}

\begin{solution}
待写
\end{solution}

\begin{question}[points = 10]
待写

% \noindent\scoringbox
\end{question}

\begin{solution}
待写
\end{solution}

\begin{question}[points = 10]
待写

% \noindent\scoringbox
\end{question}

\begin{solution}
待写
\end{solution}

\begin{question}[points = 12]
待写

% \noindent\scoringbox
\end{question}

\begin{solution}
待写
\end{solution}

% \end{xiaosihao}

\end{document}
