\makeatletter
\@ifundefined{ifShowAnswer}{%
  \newif\ifShowAnswer
}{}
\makeatother

% \ShowAnswerfalse
% \ShowAnswertrue

\examsetup{
  page = {
    size            = a4paper,
    show-columnline = true,
    foot-content    = {第~;~页~~共~;页\quad 数学分析II\quad 中国农业大学制}
  },
  solution = {
    show-solution = show-stay,
    % blank-type = manual,
    blank-type = none,
    % blank-vsep = 120ex plus 1ex minus 1ex
    blank-vsep = 15cm
  },
  fillin = {
    no-answer-type = none,
    show-answer = true
  },
  style/fullwidth-stop = catcode,
  % sealline = {
  %   show        = true,
  %   scope       = mod-2,
  %   circle-show = false,
  %   line-type   = solid,
  %   odd-info-content = {
  %     {\heiti \zihao{4}姓名} {\underline{\hspace*{8em}}},
  %     {\heiti \zihao{4}准考证号} {\examsquare{9}},
  %     {\heiti \zihao{4}考场号} {\examsquare{2}},
  %     {\heiti \zihao{4}座位号} {\examsquare{2}},
  %   },
  %   odd-info-xshift = 12mm,
  %   text = {此卷只装订不密封},
  %   text-width = 0.98\textheight,
  %   text-format  = \zihao{-3}\sffamily,
  %   text-xshift = 20mm
  % },
  question/show-points = true,
  paren = {
    show-answer = true
  },
  square = {
    x-length = 1.8em,
    y-length = 1.6em
  },
  font = times
}

\ifShowAnswer
\examsetup{
  solution/show-solution = show-stay,
  fillin/show-answer = true,
  paren/show-answer = true,
  page/size = a3paper
}
\else
\examsetup{
  solution/show-solution = hide,
  fillin/show-answer = false,
  paren/show-answer = false,
  page/size = a4paper,
  page/show-head = true,
  page/head-content = {
  \fancyhead[LE]{\xiaosihao 学院:\rule[-0.45mm]{2.5cm}{0.15mm} \hspace{0.0cm} 班级:\rule[-0.45mm]{2.5cm}{0.15mm} \hspace{0.0cm} 学号:\rule[-0.45mm]{3.5cm}{0.15mm} \hspace{0.0cm} 姓名:\rule[-0.45mm]{2.5cm}{0.15mm}}
  }
}
\fi


\ifShowAnswer
% do nothing
\else
\AtEndPreamble{%
\geometry{
left=20mm,
right=20mm,
top=20mm,
bottom=20mm,
% includehead=true,
% includefoot=true,
% heightrounded,
% showframe,% <--- just for debugging
% verbose,% <--- just for debugging
headsep=8pt
}
}
\fi


\title{
\erhao
\simli
\ifUseImageTitle
{\includegraphics[height=0.85\baselineskip]{figures/logo_cau_name.png}}\\
\else
中国农业大学\\
\fi
2024 $\sim*$ 2025学年秋季学期\\
\textbf{%
% \uline{\hspace{1.5cm}数学分析II\hspace{1.5cm}}}
\dunderline[-1pt]{1.2pt}{\hspace{1.5cm}数学分析II\hspace{1.5cm}}}
课程期末考试试题
}

\begin{document}

\maketitle

\ifShowAnswer
% do nothing
\else
\vspace{-0.9cm}

{
\begin{table}[H]
\sihao
\centering
\begin{tabular}{|wc{2cm}|wc{2cm}|wc{2cm}|wc{2cm}|wc{2cm}|wc{2.5cm}|}
\hline
题号 & 一 & 二 & 三 & 四 & 总分 \\ \hline
分数 & & & & & \\[12pt] \hline
\end{tabular}
\end{table}
}

\vspace{-0.69cm}

\begin{center}
% \textbf{\larger 全卷满分 100 分。考试用时 100 分钟。}
{\sihao (本试卷共~4~道大题)}
\end{center}

\vspace{-0.9cm}
\begin{center}
\textbf{\sihao 考生诚信承诺}
\end{center}
\vspace{-0.4cm}
% {\sihao 本人承诺自觉遵守考试纪律,诚信应考,服从监考人员管理。\\
% 本人清楚学校考试考场规则,如有违纪行为,将按照学校违纪处分规定严肃处理。}
% 注意,这里不强行超过 linewidth 的话,第二行会自动断行
\noindent\begin{minipage}[t]{1.05\linewidth}
{\sihao 本人承诺自觉遵守考试纪律,诚信应考,服从监考人员管理。\\
本人清楚学校考试考场规则,如有违纪行为,将按照学校违纪处分规定严肃处理。}
\end{minipage}

\fi


% \begin{xiaosihao}


\section{%
  选择题:本题共 5 小题, 每小题 4 分, 共 20 分。
  在每小题给出的四个选项中, 只有一项是符合题目要求的。
}

% \noindent\scoringbox

\begin{question}
  待写 \paren[]

  \begin{choices}
    \item 待写
    \item 待写
    \item 待写
    \item 待写
  \end{choices}
\end{question}

\begin{solution}
  待写
\end{solution}

\begin{question}
  待写 \paren[]

  \begin{choices}
    \item 待写
    \item 待写
    \item 待写
    \item 待写
  \end{choices}
\end{question}

\begin{solution}
  待写
\end{solution}

\begin{question}
  待写 \paren[]

  \begin{choices}
    \item 待写
    \item 待写
    \item 待写
    \item 待写
  \end{choices}
\end{question}

\begin{solution}
  待写
\end{solution}

\begin{question}
  待写 \paren[]

  \begin{choices}
    \item 待写
    \item 待写
    \item 待写
    \item 待写
  \end{choices}
\end{question}

\begin{solution}
  待写
\end{solution}

\begin{question}
  待写 \paren[]

  \begin{choices}
    \item 待写
    \item 待写
    \item 待写
    \item 待写
  \end{choices}
\end{question}

\begin{solution}
  待写
\end{solution}


\section{填空题:本题共 5 小题, 每小题 4 分, 共 20 分。}

\examsetup{
  question/index = 1
}

% \noindent\scoringbox

\begin{question}
  待写 \fillin[].
\end{question}

\begin{solution}
  待写
\end{solution}

\begin{question}
  待写 \fillin[].
\end{question}

\begin{solution}
  待写
\end{solution}

\begin{question}
  待写 \fillin[].
\end{question}

\begin{solution}
  待写
\end{solution}

\begin{question}
  待写 \fillin[].
\end{question}

\begin{solution}
  待写
\end{solution}

\begin{question}
  待写 \fillin[].
\end{question}

\begin{solution}
  待写
\end{solution}


\section{计算题:本题共 2 小题, 共 12 分。本题应写出具体演算步骤。}

\examsetup{
  question/index = 1
}

\begin{question}[points = 6]
  待写

% \noindent\scoringbox
\end{question}

\begin{solution}
  待写
\end{solution}

\begin{question}[points = 6]
  待写

% \noindent\scoringbox
\end{question}

\begin{solution}
  待写
\end{solution}


\section{解答题:本题共 5 小题, 共 48 分。解答应写出文字说明或者证明过程。}

\examsetup{
  question/index = 1
}

\begin{question}[points = 6]
  待写

% \noindent\scoringbox
\end{question}

\begin{solution}
  待写
\end{solution}

\begin{question}[points = 10]
  待写

% \noindent\scoringbox
\end{question}

\begin{solution}
  待写
\end{solution}

\begin{question}[points = 10]
  待写

% \noindent\scoringbox
\end{question}

\begin{solution}
  待写
\end{solution}

\begin{question}[points = 10]
  待写

% \noindent\scoringbox
\end{question}

\begin{solution}
  待写
\end{solution}

\begin{question}[points = 12]
  待写

% \noindent\scoringbox
\end{question}

\begin{solution}
  待写
\end{solution}

% \end{xiaosihao}

\end{document}
