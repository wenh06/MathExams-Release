\makeatletter
\@ifundefined{ifShowAnswer}{%
  \newif\ifShowAnswer
}{}
\makeatother

% \ShowAnswerfalse
% \ShowAnswertrue

\examsetup{
  page = {
    size            = a4paper,
    show-columnline = true,
    foot-content    = {第~;~页~~共~;页\quad 数学分析II\quad 中国农业大学制}
  },
  solution = {
    show-solution = show-stay,
    % blank-type = manual,
    blank-type = none,
    % blank-vsep = 120ex plus 1ex minus 1ex
    blank-vsep = 15cm
  },
  fillin = {
    no-answer-type = none,
    show-answer = true
  },
  % style/fullwidth-stop = catcode,
  style/fullwidth-stop = false,
  % sealline = {
  %   show        = true,
  %   scope       = mod-2,
  %   circle-show = false,
  %   line-type   = solid,
  %   odd-info-content = {
  %     {\heiti \zihao{4}姓名} {\underline{\hspace*{8em}}},
  %     {\heiti \zihao{4}准考证号} {\examsquare{9}},
  %     {\heiti \zihao{4}考场号} {\examsquare{2}},
  %     {\heiti \zihao{4}座位号} {\examsquare{2}},
  %   },
  %   odd-info-xshift = 12mm,
  %   text = {此卷只装订不密封},
  %   text-width = 0.98\textheight,
  %   text-format  = \zihao{-3}\sffamily,
  %   text-xshift = 20mm
  % },
  question/show-points = true,
  paren = {
    show-answer = true
  },
  square = {
    x-length = 1.8em,
    y-length = 1.6em
  },
  font = times
}

\ifShowAnswer
\examsetup{
  solution/show-solution = show-stay,
  fillin/show-answer = true,
  paren/show-answer = true,
  page/size = a3paper
}
\else
\examsetup{
  solution/show-solution = hide,
  fillin/show-answer = false,
  paren/show-answer = false,
  page/size = a4paper,
  page/show-head = true,
  page/head-content = {
  \fancyhead[LE]{\xiaosihao 学院:\rule[-0.45mm]{2.5cm}{0.15mm} \hspace{0.0cm} 班级:\rule[-0.45mm]{2.5cm}{0.15mm} \hspace{0.0cm} 学号:\rule[-0.45mm]{3.5cm}{0.15mm} \hspace{0.0cm} 姓名:\rule[-0.45mm]{2.5cm}{0.15mm}}
  }
}
\fi


\ifShowAnswer
% do nothing
\else
\AtEndPreamble{%
\geometry{
left=20mm,
right=20mm,
top=20mm,
bottom=20mm,
% includehead=true,
% includefoot=true,
% heightrounded,
% showframe,% <--- just for debugging
% verbose,% <--- just for debugging
headsep=8pt
}
}
\fi


\title{
\erhao
\simli
\ifUseImageTitle
{\includegraphics[height=0.85\baselineskip]{figures/logo_cau_name.png}}\\
\else
中国农业大学\\
\fi
2024\textasciitilde 2025学年春季学期\\
\textbf{%
% \uline{\hspace{1.5cm}数学分析II\hspace{1.5cm}}}
\dunderline[-1pt]{0.9pt}{\hspace{0.5cm}数学分析II\hspace{0.5cm}}}
\ifShowAnswer
课程第二次期中考试试题解答
\else
课程第二次期中考试试题
\fi
}

\begin{document}

\maketitle

\ifShowAnswer
% do nothing
\else
\vspace{-1.1cm}

{
\begin{table}[H]
\sihao
\centering
\begin{tabular}{|wc{2cm}|wc{2cm}|wc{2cm}|wc{2cm}|wc{2cm}|wc{2.5cm}|}
\hline
题号 & 一 & 二 & 三 & 四 & 总分 \\ \hline
分数 & & & & & \\[12pt] \hline
\end{tabular}
\end{table}
}

\vspace{-1.1cm}

\begin{center}
% \textbf{\larger 全卷满分 100 分。考试用时 100 分钟。}
{\sihao (本试卷共~4~道大题)}
\end{center}

\vspace{-1.1cm}
\begin{center}
\textbf{\sihao 考生诚信承诺}
\end{center}
\vspace{-0.5cm}
% {\sihao 本人承诺自觉遵守考试纪律,诚信应考,服从监考人员管理。\\
% 本人清楚学校考试考场规则,如有违纪行为,将按照学校违纪处分规定严肃处理。}
% 注意,这里不强行超过 linewidth 的话,第二行会自动断行
\noindent\begin{minipage}[t]{1.05\linewidth}
{\sihao 本人承诺自觉遵守考试纪律,诚信应考,服从监考人员管理。\\
本人清楚学校考试考场规则,如有违纪行为,将按照学校违纪处分规定严肃处理。}
\end{minipage}

\fi


% \begin{xiaosihao}


\section{%
  选择题:本题共 5 小题, 每小题 3 分, 共 15 分。
  在每小题给出的四个选项中, 只有一项是符合题目要求的。
}

% \noindent\scoringbox

\begin{question}
对于定义在闭区间 $[a, b]$ 上的函数, 下面哪类函数有\chemph{可能是黎曼不可积}的 \paren[D]

\begin{choices}
\item 连续函数
\item 有有限个间断点的有界函数
\item 单调有界函数
\item 简单函数 (即值域是有限集的函数)
\end{choices}
\end{question}

\begin{solution}
D 的反例: 狄利克雷函数
\end{solution}

\begin{question}
以下反常积分收敛的是 \paren[C]

\begin{choices}
\item $\displaystyle \int_0^{+\infty} \sin x ~ \mathrm{d} x$
\item $\displaystyle \int_0^{+\infty} |\sin x| ~ \mathrm{d} x$
\item $\displaystyle \int_0^{+\infty} \sin (x^2) ~ \mathrm{d} x$
\item $\displaystyle \int_0^{+\infty} \sin^2 x ~ \mathrm{d} x$
\end{choices}
\end{question}

\begin{solution}
A, B, D 都可以通过在一个最小正周期上的积分值判断出不可积.

对于 C, 有
$$\int_0^{+\infty} \sin (x^2) ~ \mathrm{d} x = \int_0^1 \sin (x^2) ~ \mathrm{d} x + \int_1^{+\infty} \dfrac{\sin (u)}{2\sqrt{u}} ~ \mathrm{d} u,$$
后者可以通过狄利克雷判别法知道是收敛的.
\end{solution}

\begin{question}
以下的论断\chemph{正确}的是 \paren[D]

\begin{choices}
\item 若闭区间 $[a, b]$ 上黎曼可积函数列 $\{ f_n(x) \}_{n \in \mathbb{N}}$ 点态收敛到 $[a, b]$ 上黎曼可积函数 $f(x),$ 但不是一致收敛, 则必然有 $\displaystyle \lim_{n\to\infty} \int_a^b f_n(x) ~ \mathrm{d}x \neq \int_a^b f(x) ~ \mathrm{d}x$
\item 设级数 $\displaystyle \sum_{n=1}^{\infty} x_n$ 收敛, 数列 $\{ \alpha_n \}_{n \in \mathbb{N}}$ 满足 $\displaystyle \lim_{n \to \infty} \alpha_n = 0,$ 则级数 $\displaystyle \sum_{n=1}^{\infty} \alpha_n x_n$ 必定收敛
\item 设 $f(x), g(x)$ 是定义在 $[0, +\infty)$ 上的连续函数. 假设反常积分 $\displaystyle \int_0^{+\infty} f(x) ~ \mathrm{d} x$ 收敛, 并且有 $\displaystyle \lim_{n\to+\infty} g(x) = 0,$ 则反常积分 $\displaystyle \int_0^{+\infty} f(x)g(x) ~ \mathrm{d} x$ 必定收敛
\item 以上说法都不对
\end{choices}
\end{question}

\begin{solution}
A 的反例: 任何一个一致有界的黎曼可积函数列 $\{ f_n(x) \}_{n \in \mathbb{N}}$ 都是反例.

B 的反例: $\displaystyle x_n = (-1)^n \dfrac{1}{n + 1}; \quad \alpha_n = (-1)^n \dfrac{1}{\ln^q (n + 1)}, ~ 0 < q \leqslant 1.$

C 的反例: $\displaystyle f(x) = \dfrac{\sin x}{x}; \quad g(x) = \begin{cases}
\dfrac{\sin e}{e} x, & 0 \leqslant x \leqslant e, \\
\dfrac{\sin x}{\ln^q x}, & x > e,
\end{cases} \quad 0 < q \leqslant 1.$
\end{solution}

\begin{question}
设数项级数 $\displaystyle \sum_{n=1}^{\infty} a_n$ 收敛, $a_n \neq -1,$ 但 $\displaystyle \sum_{n=1}^{\infty} a_n^2$ 发散, 则以下说法正确的是 \paren[A]

\begin{choices}
\item 级数 $\displaystyle \sum_{n=1}^{\infty} \dfrac{a_n}{n + a_n^2}$ 必然收敛
\item 级数 $\displaystyle \sum_{n=1}^{\infty} |a_n|$ 可能收敛也可能发散
\item 无穷乘积 $\displaystyle \prod_{n=1}^{\infty} (1 + a_n)$ 可能收敛也可能发散
\item 无穷乘积 $\displaystyle \prod_{n=1}^{\infty} (1 + a_n^2)$ 可能收敛也可能发散
\end{choices}
\end{question}

\begin{solution}
由 $\displaystyle \sum_{n=1}^{\infty} a_n$ 收敛知 $a_n \to 0.$ 于是当 $n$ 充分大时, $\displaystyle \dfrac{1}{n + a_n^2}$ 是单调递减趋于 $0$ 的, 于是由 A-D 判别法知 $\displaystyle \sum_{n=1}^{\infty} \dfrac{a_n}{n + a_n^2}$ 必然收敛.

由于 $a_n \to 0,$ 那么当 $n$ 充分大时, $|a_n| < 1,$ 从而有 $|a_n| > |a_n| \cdot |a_n| = a_n^2,$ 由正项级数的比较判别法知 $\displaystyle \sum_{n=1}^{\infty} |a_n|$ 必然发散.

在 $\displaystyle \sum_{n=1}^{\infty} a_n$ 收敛的前提下, $\displaystyle \sum_{n=1}^{\infty} a_n^2$ 与 $\displaystyle \prod_{n=1}^{\infty} (1 + a_n)$ 敛散性相同, 故 $\displaystyle \prod_{n=1}^{\infty} (1 + a_n)$ 必然是发散的.

$\displaystyle \sum_{n=1}^{\infty} a_n^2$ 为正项级数, 它的敛散性与 $\displaystyle \prod_{n=1}^{\infty} (1 + a_n^2)$ 相同, 故 $\displaystyle \prod_{n=1}^{\infty} (1 + a_n^2)$ 必然是发散的.
\end{solution}

\begin{question}
设幂级数 $\sum\limits_{n=0}^\infty a_n x^n$ 满足上极限 $\varlimsup_{n\to\infty} \left\lvert \dfrac{a_{n+1}}{a_n} \right\rvert = A,$ $0 < A < +\infty,$ 那么下面正确的论断是 \paren[B]

\begin{choices}
\item $\sum\limits_{n=0}^\infty a_n x^n$ 的收敛半径必定等于 $\dfrac{1}{A}$
\item $\sum\limits_{n=0}^\infty a_n x^n$ 的收敛半径可能大于 $\dfrac{1}{A}.$
\item $\sum\limits_{n=0}^\infty a_n x^n$ 的收敛半径可能小于 $\dfrac{1}{A}.$
\item 以上说法都不对
\end{choices}
\end{question}

\begin{solution}
有不等式
$$\varliminf_{n\to\infty} \left\lvert \dfrac{a_{n+1}}{a_n} \right\rvert \leqslant \varliminf_{n\to\infty} \nroot[n]{\lvert a_n \rvert} \leqslant \varlimsup_{n\to\infty} \nroot[n]{\lvert a_n \rvert} \leqslant \varlimsup_{n\to\infty}  \left\lvert \dfrac{a_{n+1}}{a_n} \right\rvert = A,$$
而收敛半径等于 $1 / \varlimsup_{n\to\infty} \nroot[n]{\lvert a_n \rvert},$ 因此而收敛半径大于 $\dfrac{1}{A},$ 等于 $\dfrac{1}{A}$ 都是可能的.

等于 $\dfrac{1}{A}$ 的例子: $a_n = \dfrac{1}{n},$ 那么 $1 = \varliminf_{n\to\infty} \left\lvert \dfrac{a_{n+1}}{a_n} \right\rvert = \varlimsup_{n\to\infty} \nroot[n]{\lvert a_n \rvert}.$

大于 $\dfrac{1}{A}$ 的例子: $a_n = \dfrac{3^{n}}{5^{n + (n \mod 2)}},$ 那么 $3 = \varlimsup_{n\to\infty} \left\lvert \dfrac{a_{n+1}}{a_n} \right\rvert > \varlimsup_{n\to\infty} \nroot[n]{\lvert a_n \rvert} = \dfrac{3}{5}.$
\end{solution}


\section{填空题:本题共 5 小题, 每小题 3 分, 共 15 分。}

\examsetup{
  question/index = 1
}

% \noindent\scoringbox

\begin{question}
求柯西主值 $\displaystyle (\operatorname{cpv})\int_{-\infty}^{+\infty} \cos x ~ \mathrm{d}x =$ \fillin[0].
\end{question}

\begin{solution}
直接计算极限 $\displaystyle \lim_{c\to+\infty} \int_{-c}^{c} \cos x ~ \mathrm{d}x = 0.$
\end{solution}

\begin{question}
已知 $\displaystyle \int_0^{+\infty} \dfrac{\sin x}{x} \mathrm{d}x = \dfrac{\pi}{2},$ 那么 $\displaystyle \int_0^{+\infty} \dfrac{\sin^2 x}{x^2} \mathrm{d}x =$ \fillin[$\dfrac{\pi}{2}$].
\end{question}

\begin{solution}
容易判断 $\displaystyle \int_0^{+\infty} \dfrac{\sin^2 x}{x^2} \mathrm{d}x$ 收敛. 由分部积分法有
\begin{align*}
\int_0^{+\infty} \dfrac{\sin^2 x}{x^2} \mathrm{d}x & = - \int_0^{+\infty} \sin^2 x ~\mathrm{d} \left(\frac{1}{x}\right) = - \dfrac{\sin^2 x}{x} \bigg|_0^{+\infty} + \int_0^{+\infty} \dfrac{2\sin x \cos x}{x} \mathrm{d}x \\
& = \int_0^{+\infty} \dfrac{\sin 2x}{2x} \mathrm{d}(2x) = \int_0^{+\infty} \dfrac{\sin x}{x} \mathrm{d}x = \dfrac{\pi}{2}
\end{align*}
\end{solution}

\begin{question}
若正项级数 $\displaystyle \sum_{n=2}^\infty \dfrac{1}{n \ln^q n}$ 收敛, 则实数 $q$ 的取值范围为 \fillin[$q > 1$].
\end{question}

\begin{solution}
直接由正项级数的积分判别法进行判断.
\end{solution}

\begin{question}
幂级数 $\displaystyle \sum\limits_{n=1}^{\infty} \dfrac{\left( 2 + (-1)^n \right)^n}{2^n + \sin nx} x^n$ 的收敛半径等于 \fillin[$\dfrac{2}{3}$].
\end{question}

\begin{solution}
记 $\displaystyle a_n = \dfrac{\left( 2 + (-1)^n \right)^n}{2^n + \sin nx} > 0,$ 那么 $\displaystyle \varlimsup_{n\to\infty} \nroot[n]{a_n} = \dfrac{3}{2},$ 所以收敛半径等于 $\dfrac{2}{3}$
\end{solution}

\begin{question}
函数项级数 $\displaystyle \sum_{n=1}^{\infty} \dfrac{\sin nx}{n}$ 的收敛域为 \fillin[$\mathbb{R}$].
\end{question}

\begin{solution}
由于 $x \not\in 2\pi\mathbb{Z}$ 时有 (见 (下册) 课本例 9.4.2)
\begin{equation*}
2\sin\dfrac{x}{2} \sum_{n=1}^{\infty} \sin nx = \cos\dfrac{x}{2} - \cos\dfrac{2n+1}{2}x,
\end{equation*}
而$\frac{1}{n} \searrow 0,$ 由狄利克雷判别法知收敛. 对于 $x \in 2\pi\mathbb{Z},$ 原级数每一项都是 $0,$ 从而也是收敛的.
\end{solution}


\section{计算题:本题共 2 小题, 共 20 分。本题应写出具体演算步骤。}

\examsetup{
  question/index = 1
}

\begin{question}[points = 10]
请计算椭圆 $\displaystyle \dfrac{x^2}{a^2} + \dfrac{y^2}{b^2} = 1, ~ a > b > 0,$ 绕 $x$ 轴旋转一周形成的椭球的表面积.

% \noindent\scoringbox
\end{question}

\begin{solution}
用参数方程形式求解: $\begin{cases} x(t) = a \sin t, \\ y(t) = b \cos t,\end{cases}$ $t \in \left[ -\frac{\pi}{2}, \frac{\pi}{2} \right],$ 那么表面积 $S$ 有
\begin{align*}
S & = 2\pi \int_{-\frac{\pi}{2}}^{\frac{\pi}{2}} y(t) \sqrt{(x'(t))^2 + (y'(t))^2} ~ \mathrm{d} t = 2\pi \int_{-\frac{\pi}{2}}^{\frac{\pi}{2}} b \cos t \sqrt{a^2 \cos^2 t + b^2 \sin^2 t} ~ \mathrm{d} t \\
& = 2 \pi ab \int_{-\frac{\pi}{2}}^{\frac{\pi}{2}} \cos t \sqrt{1 - k^2\sin^2 t} ~ \mathrm{d} t = \dfrac{2 \pi ab}{k} \int_{-\frac{\pi}{2}}^{\frac{\pi}{2}} \sqrt{1 - k^2\sin^2 t} ~ \mathrm{d} (k \sin t) \\
& = \dfrac{2 \pi ab}{k} \int_{-k}^{k} \sqrt{1 - u^2} ~ \mathrm{d} u = \dfrac{2 \pi ab}{k} \left( u\sqrt{1-u^2} + \arcsin u \right) \bigg|_0^k \\
& = \dfrac{2 \pi ab}{k} \left( k\sqrt{1-k^2} + \arcsin k \right) \\
& = 2\pi b^2 + \dfrac{2\pi a^2b}{\sqrt{a^2 - b^2}} \arcsin\dfrac{\sqrt{a^2 - b^2}}{a}.
\end{align*}
上式中的 $k = \dfrac{\sqrt{a^2 - b^2}}{a}$ 为离心率.
\end{solution}

\begin{question}[points = 10]
求定积分 $\displaystyle \int_{-\pi/2}^{\pi/2} \dfrac{\cos x}{1 + 2^x} ~ \mathrm{d}x.$ (提示: 将积分区间分为对称的两部分).

% \noindent\scoringbox
\end{question}

\begin{solution}
\begin{align*}
\int_{-\pi/2}^{\pi/2} \dfrac{\cos x}{1 + 2^x} ~ \mathrm{d}x & = \int_{-\pi/2}^{0} \dfrac{\cos x}{1 + 2^x} ~ \mathrm{d}x + \int_{0}^{\pi/2} \dfrac{\cos x}{1 + 2^x} ~ \mathrm{d}x \\
& = \int_{0}^{\pi/2} \dfrac{\cos (-x)}{1 + 2^{-x}} ~ \mathrm{d}x + \int_{0}^{\pi/2} \dfrac{\cos x}{1 + 2^x} ~ \mathrm{d}x \\
& = \int_{0}^{\pi/2} \dfrac{2^{x} \cos x}{1 + 2^{x}} ~ \mathrm{d}x + \int_{0}^{\pi/2} \dfrac{\cos x}{1 + 2^x} ~ \mathrm{d}x \\
& = \int_{0}^{\pi/2} \dfrac{(1 + 2^{x}) \cos x}{1 + 2^{x}} ~ \mathrm{d}x \\
& = \int_{0}^{\pi/2} \cos x ~ \mathrm{d}x = 1.
\end{align*}
\end{solution}


\section{解答题:本题共 5 小题, 共 50 分。解答应写出文字说明或者证明过程。}

\examsetup{
  question/index = 1
}

\begin{question}[points = 8]
设函数 $f(x)$ 是 $[a, b]$ 区间上的\chemph{非负}连续函数, 并且满足 $\displaystyle \int_a^b f(x) ~ \mathrm{d}x = 0,$ 请证明 $f(x)$ 在 $[a, b]$ 上恒等于 $0.$

% \noindent\scoringbox
\end{question}

\begin{solution}
本题是 (上册) 课本 \S 7.2 习题 5 的等价变形.

用反证法. 假设非负函数 $f(x)$ 不恒等于 $0,$ 即存在 $x_0 \in [a,b],$ 使得 $f(x_0) > 0.$ 由 $f(x)$ 的连续性知存在 $x_0$ 在 $[a, b]$ 中的非平凡闭邻域 $[c, d],$ 满足 $x_0 \in [c, d] \subset* [a, b],$ $c < d,$ 且 $\forall ~ x \in [c, d],$ 有 $f(x) > \frac{f(x_0)}{2}.$ 于是
\begin{equation*}
\int_a^b f(x) ~ \mathrm{d}x \geqslant \int_c^d f(x) ~ \mathrm{d}x \geqslant \int_c^d \frac{f(x_0)}{2} ~ \mathrm{d}x = \frac{f(x_0)}{2}(d - c) > 0,
\end{equation*}
这与已知条件 $\displaystyle \int_a^b f(x) ~ \mathrm{d}x = 0$ 矛盾.
\end{solution}


\begin{question}[points = 10]
设无穷乘积 $\displaystyle \prod_{n=1}^\infty (1+a_n)$ 收敛, 请问数项级数 $\displaystyle \sum_{n=1}^\infty a_n$ 是否必收敛? 若是, 请证明这个结论; 若否, 请给出反例.

% \noindent\scoringbox
\end{question}

\begin{solution}
不一定.

可以举 (下册) 课本 \S 9.5 习题 7 中的反例:
\begin{equation*}
a_1 = 0, \quad a_{2n+1} = -\dfrac{1}{\sqrt{n}}, \quad a_{2n} = \dfrac{1}{\sqrt{n}} + \dfrac{1}{n} \left( 1 + \dfrac{1}{\sqrt{n}} \right)
\end{equation*}
\end{solution}


\begin{question}[points = 10]
设函数 $S(x)$ 在闭区间 $[0,1]$ 上连续, 且 $S(1) = 0.$ 请证明: $\{x^n S(x)\}$ 在 $[0,1]$ 上一致收敛.

% \noindent\scoringbox
\end{question}

\begin{solution}
本题是 (下册) 课本 \S 10.1 习题 8

由于 $S(x)$ 在闭区间 $[0,1]$ 上连续, 那么 $S(x)$ 在 $[0,1]$ 上有界, 可以设 $M$ 为 $|S(x)|$ 的一个上界. 又由于 $S(1) = 0,$ 所以 $\forall ~ \varepsilon > 0,$ $\exists ~ \delta > 0,$ 使得 $\forall ~ x \in (1-\delta, 1]$ 有 $|S(x)| < \varepsilon,$ 从而有
\begin{equation*}
|x^nS(x)| \leqslant |S(x)| < \varepsilon, \quad \forall ~ x \in (1-\delta, 1], ~ \forall ~ n \in \mathbb{N}.
\end{equation*}
另一方面, 取 $N = \left\lceil \log_{1-\delta} \left( \frac{\varepsilon}{M} \right) + 1 \right\rceil,$ 那么在区间 $[0, 1 - \delta]$ 上有
\begin{equation*}
|x^nS(x)| \leqslant (1 - \delta)^n M < \varepsilon, \quad \forall~x \in [0, 1 - \delta], ~ \forall n > N.
\end{equation*}
综上即有 $\forall ~ n > N = \left\lceil \log_{1-\delta} \left( \frac{\varepsilon}{M} \right) + 1 \right\rceil,$ $\forall ~ x \in [0, 1]$ 有 $|x^nS(x)| < \varepsilon,$ 这就证明了 $\{x^n S(x)\}$ 在 $[0,1]$ 上一致收敛到常值函数 $0.$
\end{solution}


\begin{question}[points = 10]
设 $\displaystyle f(x) = \sum\limits_{n=1}^{\infty} \dfrac{\cos nx}{\sqrt{n^3 + n}},$
\begin{enumerate}
\item 求证: $f(x)$ 在 $\mathbb{R}$ 上连续;
\item 记 $\displaystyle F(x) = \int_0^x f(t) ~ \mathrm{d} t,$ 求证: $\dfrac{\sqrt{2}}{2} - \dfrac{1}{15} < F \left( \dfrac{\pi}{2} \right) < \dfrac{\sqrt{2}}{2}.$
\end{enumerate}

% \noindent\scoringbox
\end{question}

\begin{solution}

本题是 (下册) 课本 §10.2 习题 12

\begin{enumerate}
\item 由于 $\displaystyle \left\lvert \dfrac{\cos nx}{\sqrt{n^3 + n}} \right\rvert \leqslant \dfrac{1}{\sqrt{n^3 + n}} < n^{-3/2},$ 而正项级数 $\displaystyle \sum\limits_{n=1}^{\infty} n^{-3/2}$ 收敛, 所以由 Weierstraß 判别法知函数项级数 $\displaystyle f(x) = \sum\limits_{n=1}^{\infty} \dfrac{\cos nx}{\sqrt{n^3 + n}}$ 一致收敛. 由一致收敛函数项级数的连续性定理知, 和函数连续.
\item 由一致收敛函数项级数的逐项积分定理知
\begin{align*}
F(x) & = \int_0^x f(t) ~ \mathrm{d} t = \int_0^x \sum\limits_{n=1}^{\infty} \dfrac{\cos nt}{\sqrt{n^3 + n}} ~ \mathrm{d} t = \sum\limits_{n=1}^{\infty} \int_0^x \dfrac{\cos nt}{\sqrt{n^3 + n}} ~ \mathrm{d} t \\
& = \sum\limits_{n=1}^{\infty} \left. \dfrac{\sin nt}{n\sqrt{n^3 + n}} \right|_0^x = \sum\limits_{n=1}^{\infty} \dfrac{\sin nx}{n\sqrt{n^3 + n}},
\end{align*}
将 $x = \dfrac{\pi}{2}$ 代入上式得
\begin{align*}
F \left( \dfrac{\pi}{2} \right) & = \sum\limits_{n=1}^{\infty} \dfrac{\sin \dfrac{n\pi}{2}}{n\sqrt{n^3 + n}} = \sum\limits_{k=0}^{\infty} \dfrac{(-1)^k}{(2k+1)\sqrt{(2k+1)^3 + (2k+1)}} = : \sum\limits_{k=0}^{\infty} (-1)^k a_k
\end{align*}
上述关于 $k$ 的级数是 Leibniz 级数, 而且 $a_k$ 是严格单调递减的, 所以
\begin{align*}
F \left( \dfrac{\pi}{2} \right) & < a_0 = \dfrac{1}{\sqrt{2}} = \dfrac{\sqrt{2}}{2}, \\
F \left( \dfrac{\pi}{2} \right) & > a_0 - a_1 = \dfrac{\sqrt{2}}{2} - \dfrac{1}{3 \cdot \sqrt{3^3 + 3}} > \dfrac{\sqrt{2}}{2} - \dfrac{1}{15}.
\end{align*}
\end{enumerate}

\end{solution}


\begin{question}[points = 12]
考虑定义在 $\mathbb{R}$ 上的函数 $\varphi(x) = d(x, \mathbb{Z}) = \min\limits_{n \in \mathbb{Z}} |x - n|.$ $\varphi(x)$ 是周期为 $1$ 的函数, 在 $[0, 1)$ 区间上的定义为
$$
\varphi(x) = \begin{cases}
x, & 0 \leqslant x < 1/2, \\ 1 - x, & 1/2 \leqslant x < 1.
\end{cases}
$$
定义 Generalized Van der Waerden-Takagi 函数为
$$
f(x) = \sum_{n=0}^{\infty} a^n \varphi(b^n x),
$$
其中 $a,b \in \mathbb{R}$ 为常数.
\begin{enumerate}
\item 若 $0 < a < 1,$ 请证明: $f(x)$ 在 $\mathbb{R}$ 上连续.
\item 设 $g(x)$ 为定义在 $\mathbb{R}$ 上任一函数, 请证明: 若 $g(x)$ 在某点 $x_0$ 处可导, 导数值等于 $A,$ 那么对于任意的序列 $u_n, v_n,$ 若满足 $u_n \leqslant x_0 \leqslant v_n,$ 且 $\displaystyle \lim_{n\to\infty} u_n = \lim_{n\to\infty} v_n = x_0,$ 那么必有
$$
\displaystyle \lim_{n\to\infty} \dfrac{g(u_n) - g(v_n)}{u_n - v_n} = A.
$$
\item 若 $0 < a < 1,$ $b \in 2 \mathbb{N}$ 为正偶数, 且满足 $ab \geqslant 1,$ 请证明: $f(x)$ 在 $\mathbb{R}$ 中任意点处都不可导.
\end{enumerate}

% \noindent\scoringbox
\end{question}

\begin{solution}
\begin{enumerate}
\item 由于 $0 \leqslant \varphi(x) \leqslant 1/2,$
所以 $0 \leqslant a^n \varphi(b^n x) \leqslant a^n/2.$ 由于正项级数 $\displaystyle \sum_{n=0}^{\infty} a^n$ 收敛,
于是由 Weierstraß 判别法, 函数项级数 $\displaystyle \sum_{n=0}^{\infty} a^n \varphi(b^n x)$ 一致收敛.
进一步由一致收敛函数项级数和函数的连续性定理知, 和函数 $f(x)$ 在 $\mathbb{R}$ 上连续.
\item 记 $h_{1,n} = x_0 - u_n,$ $h_{2,n} = v_n - x_0,$ 那么 $h_{1,n}, h_{2,n} \geqslant 0$ 且 $h_{1,n} \to 0,$ $h_{2,n} \to 0.$ 由于 $g(x)$ 在点 $x_0$ 处可导, 所以在 $x_0$ 附近有
\begin{equation*}
g(x) = g(x_0) + A (x - x_0) + o(|x - x_0|),
\end{equation*}
那么
\begin{align*}
\lim_{n\to\infty} \dfrac{g(u_n) - g(v_n)}{u_n - v_n} & = \lim_{n\to\infty} \dfrac{A (h_{1,n} + h_{2,n}) + o(h_{1,n}) + o(h_{1,n})}{h_{1,n} + h_{2,n}} \\
& = A + \lim_{n\to\infty} \dfrac{o(h_{1,n}) + o(h_{1,n})}{h_{1,n} + h_{2,n}} \\
& = A + 0 = A.
\end{align*}
\item 由于 $\varphi(x)$ 是周期为 $1$ 的函数, 所以 $f$ 也以 $1$ 为周期, 所以只要对任意 $x \in [0, 1)$ 证明 $f$ 在 $x$ 处不可导即可. 由第 (2) 问, 希望可以选取序列 $u_n \rightarrow x \leftarrow v_n,$ 使得
\begin{equation*}
\dfrac{f(u_n) - f(v_n)}{u_n - v_n}
= \dfrac{\sum\limits_{m=0}^{\infty} a^m (\varphi(b^m u_n) - \varphi(b^m v_n))}{u_n - v_n}
\end{equation*} 
发散. 由于 $2 b^n x$ 是非负实数, 所以存在非负整数 $k_n,$ 使得 $k_n \leqslant 2 b^n x < k_n + 1,$ 即
\begin{equation*}
\dfrac{k_n}{2 b^n} \leqslant x < \dfrac{k_n + 1}{2 b^n}.
\end{equation*}
取 $u_n = \dfrac{k_n}{2 b^n}, v_n = \dfrac{k_n + 1}{2 b^n}.$ 对于和式 $\displaystyle \sum\limits_{m=0}^{\infty} a^m (\varphi(b^m u_n) - \varphi(b^m v_n)),$ 分为两部分讨论:

$1^{\circ}$ 当 $0 \leqslant m < n$ 时, 由于
\begin{equation*}
b^m u_n = \dfrac{k_n}{2} \cdot \dfrac{1}{b^{n-m}}, ~~ b^m v_n = \dfrac{k_n + 1}{2} \cdot \dfrac{1}{b^{n-m}},
\end{equation*}
从而存在非负整数 $z \in \mathbb{Z}_{\geqslant 0},$ 使得 $b^m u_n, b^m v_n$ 同时属于 $[z, z+1/2]$ 或 $[z+1/2, z+1],$ 那么
\begin{equation*}
\varphi(b^m u_n) - \varphi(b^m v_n) = \pm (b^m u_n - b^m v_n) = \pm \dfrac{1}{2 b^{n-m}},
\end{equation*}
从而有
\begin{equation*}
\dfrac{a^m (\varphi(b^m u_n) - \varphi(b^m v_n))}{u_n - v_n}
= \pm a^m \dfrac{1}{2 b^{n-m}} \left/ \left( \dfrac{1}{2 b^n}\right) \right.
= \pm (ab)^m.
\end{equation*}

$2^{\circ}$ 当 $m \geqslant n$ 时, 有
\begin{equation*}
b^m u_n = \dfrac{b^{m-n} \cdot k_n}{2}, ~~ b^m v_n = \dfrac{b^{m-n} \cdot (k_n+1)}{2}.
\end{equation*}
由于 $b$ 为偶数, 若 $m > n,$ 则 $b^m u_n, b^m v_n$ 都是整数;
若 $m = n,$ 则 $b^m u_n = \dfrac{k_n}{2}, b^m v_n = \dfrac{k_n + 1}{2}$ 其中一个为整数,
另一个为半整数 (即 $\dfrac{1}{2} +$ 整数). 于是
\begin{equation*}
\dfrac{a^m (\varphi(b^m u_n) - \varphi(b^m v_n))}{u_n - v_n}
= \begin{cases}
    0, & \text{当 } m > n \text{ 时}, \\
    \dfrac{\pm a^m / 2}{1 / (2 b^{m})} = \pm (ab)^m, & \text{当 } m = n \text{ 时}.
\end{cases}
\end{equation*}
于是
\begin{equation*}
\dfrac{f(u_n) - f(v_n)}{u_n - v_n}
= \dfrac{\sum\limits_{m=0}^{\infty} a^m (\varphi(b^m u_n) - \varphi(b^m v_n))}{u_n - v_n}
= \sum\limits_{m=1}^n \left( \pm (ab)^m \right).
\end{equation*}
由于 $ab \geqslant 1,$ 所以 $\displaystyle \lim\limits_{n\to\infty} \sum\limits_{m=1}^n \left( \pm (ab)^m \right) = \sum\limits_{m=1}^{\infty} \left( \pm (ab)^m \right)$ 发散.
\end{enumerate}
\end{solution}

% \end{xiaosihao}

\end{document}
