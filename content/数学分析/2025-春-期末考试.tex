\makeatletter
\@ifundefined{ifShowAnswer}{%
  \newif\ifShowAnswer
}{}
\makeatother

% \ShowAnswerfalse
% \ShowAnswertrue

\examsetup{
  page = {
    size            = a4paper,
    show-columnline = true,
    foot-content    = {第~;~页~~共~;页\quad 数学分析II\quad 中国农业大学制}
  },
  solution = {
    show-solution = show-stay,
    % blank-type = manual,
    blank-type = none,
    % blank-vsep = 120ex plus 1ex minus 1ex
    blank-vsep = 15cm
  },
  fillin = {
    no-answer-type = none,
    show-answer = true
  },
  % style/fullwidth-stop = catcode,
  style/fullwidth-stop = false,
  % sealline = {
  %   show        = true,
  %   scope       = mod-2,
  %   circle-show = false,
  %   line-type   = solid,
  %   odd-info-content = {
  %     {\heiti \zihao{4}姓名} {\underline{\hspace*{8em}}},
  %     {\heiti \zihao{4}准考证号} {\examsquare{9}},
  %     {\heiti \zihao{4}考场号} {\examsquare{2}},
  %     {\heiti \zihao{4}座位号} {\examsquare{2}},
  %   },
  %   odd-info-xshift = 12mm,
  %   text = {此卷只装订不密封},
  %   text-width = 0.98\textheight,
  %   text-format  = \zihao{-3}\sffamily,
  %   text-xshift = 20mm
  % },
  question/show-points = true,
  paren = {
    show-answer = true
  },
  square = {
    x-length = 1.8em,
    y-length = 1.6em
  },
  font = times
}

\ifShowAnswer
\examsetup{
  solution/show-solution = show-stay,
  fillin/show-answer = true,
  paren/show-answer = true,
  page/size = a3paper
}
\else
\examsetup{
  solution/show-solution = hide,
  fillin/show-answer = false,
  paren/show-answer = false,
  page/size = a4paper,
  page/show-head = true,
  page/head-content = {
  \fancyhead[LE]{\xiaosihao 学院:\rule[-0.45mm]{2.5cm}{0.15mm} \hspace{0.0cm} 班级:\rule[-0.45mm]{2.5cm}{0.15mm} \hspace{0.0cm} 学号:\rule[-0.45mm]{3.5cm}{0.15mm} \hspace{0.0cm} 姓名:\rule[-0.45mm]{2.5cm}{0.15mm}}
  }
}
\fi


\ifShowAnswer
% do nothing
\else
\AtEndPreamble{%
\geometry{
left=20mm,
right=20mm,
top=20mm,
bottom=20mm,
% includehead=true,
% includefoot=true,
% heightrounded,
% showframe,% <--- just for debugging
% verbose,% <--- just for debugging
headsep=8pt
}
}
\fi


\title{
\erhao
\simli
\ifUseImageTitle
{\includegraphics[height=0.85\baselineskip]{figures/logo_cau_name.png}}\\
\else
中国农业大学\\
\fi
2024\textasciitilde 2025学年春季学期\\
\textbf{%
% \uline{\hspace{1.5cm}数学分析II\hspace{1.5cm}}}
\dunderline[-1pt]{0.9pt}{\hspace{0.5cm}数学分析II\hspace{0.5cm}}}
\ifShowAnswer
课程考试试题解答
\else
课程考试试题
\fi
}

\begin{document}

\maketitle

\ifShowAnswer
% do nothing
\else
\vspace{-1.1cm}

{
\begin{table}[H]
\sihao
\centering
\begin{tabular}{|wc{2cm}|wc{2cm}|wc{2cm}|wc{2cm}|wc{2cm}|wc{2.5cm}|}
\hline
题号 & 一 & 二 & 三 & 四 & 总分 \\ \hline
分数 & & & & & \\[12pt] \hline
\end{tabular}
\end{table}
}

\vspace{-1.1cm}

\begin{center}
% \textbf{\larger 全卷满分 100 分。考试用时 100 分钟。}
{\sihao (本试卷共~4~道大题)}
\end{center}

% \vspace{-1.1cm}
\begin{center}
\textbf{\sihao 考生诚信承诺}
\end{center}
\vspace{-0.5cm}
% {\sihao 本人承诺自觉遵守考试纪律,诚信应考,服从监考人员管理。\\
% 本人清楚学校考试考场规则,如有违纪行为,将按照学校违纪处分规定严肃处理。}
% 注意,这里不强行超过 linewidth 的话,第二行会自动断行
\noindent\begin{minipage}[t]{1.05\linewidth}
{\sihao 本人承诺自觉遵守考试纪律,诚信应考,服从监考人员管理。\\
本人清楚学校考试考场规则,如有违纪行为,将按照学校违纪处分规定严肃处理。}
\end{minipage}

\fi


% \begin{xiaosihao}


\section{%
  选择题:本题共 5 小题, 每小题 3 分, 共 15 分。
  在每小题给出的四个选项中, 只有一项是符合题目要求的。
}

% \noindent\scoringbox

\begin{question}
以下集合是紧集的是 \paren[A]

\begin{choices}
\item $\{ (x_1, \dots, x_n) \in \mathbb{R}^n ~ : ~ |x_1| + \cdots + |x_n| = 1 \}$
\item $\{ (x_1, \dots, x_n) \in \mathbb{R}^n ~ : ~ x_1, \dots, x_n \in \mathbb{Q} \cap* [-1, 1] \}$
\item $\displaystyle \bigcup_{n=1}^{\infty} [n, n + 1/2]$
\item $\displaystyle \left\{ \dfrac{1}{n} ~ : ~ n \in \mathbb{N}_+ \right\}$
\end{choices}
\end{question}

\begin{solution}
A 是 1-范数下的单位球面, 是有界闭集, 从而是紧集. 它是闭集是因为
$$
\| \cdot \|_1: \mathbb{R}^n \rightarrow \mathbb{R}, ~ (x_1, \dots, x_n) \mapsto |x_1| + \cdots + |x_n|
$$
是连续函数, 单位球面正好是闭集 $\{1\}$ 的原像, 从而是闭集.

B 是有界集, 但不是闭集, 例如 $(\sqrt{2}/2, 0, \dots, 0)$ 是这个集合的一个极限点, 但不属于这个集合.

C 虽然是闭集, 但不是有界集.

D 不是闭集, 因为 $0$ 是它的极限点, 但不在该集合内.
\end{solution}


\begin{question}
以下是道路连通集的是 \paren[D]

\begin{choices}
\item $\{ (0,y) ~:~ -1\leqslant y \leqslant 1\} \cup* \{ (x, \sin(1/x)) ~:~ 0 < x < 1 \}$
\item $n$ 可逆阶实系数方阵全体 $\operatorname{GL}_n(\mathbb{R}) = \{ A \in M_{n \times n}(\mathbb{R}) ~:~ \det A \neq 0 \}$
\item $n$ 阶实系数正交方阵全体 $\operatorname{O}_n(\mathbb{R}) = \{ A \in M_{n \times n}(\mathbb{R}) ~:~ A^TA = AA^T = I_n \}$
\item 行列式等于 $1$ 的 $n$ 阶实系数方阵全体 $\operatorname{SL}_n(\mathbb{R}) = \{ A \in M_{n \times n}(\mathbb{R}) ~:~ \det A = 1 \}$
\end{choices}
\end{question}

\begin{solution}
A 是拓扑意义下连通但不是道路连通的集合

B, C 不是道路连通集原因类似, 因为 $\det$ 是连续函数, 而 $\operatorname{GL}_n(\mathbb{R})$ 与 $\operatorname{O}_n(\mathbb{R})$ 中方阵的行列式有的是正的有的是负的.
\end{solution}


\begin{question}
以下说法\chemph{正确}的是 \paren[B]

\begin{choices}
\item 设 $E$ 是 $\mathbb{R}^n$ 中的非空点集, $x \in \mathbb{R}^n$ 为一点, 若存在\chemph{点列} (序列) $x_n \in E, n = 1, 2, \dots,$ 使得 $\lim\limits_{n\to\infty} x_n = x,$ 则 $x$ 是集合 $E$ 的聚点.
\item 设 $f$ 是定义在 $\mathbb{R}^n$ 上取值在 $\mathbb{R}^m$ 中的连续向量值函数, $E$ 是 $\mathbb{R}^n$ 中的非空有界闭集, 那么 $E$ 在 $f$ 下的像集 $f(E)$ 必然是 $\mathbb{R}^m$ 中的闭集.
% \item 设 $\{f_n(x)\}_{n\in\mathbb{N}}$ 是定义在闭区间 $D = [a, b]$ 上的函数列, 在 $D$ 上点态收敛到函数 $f(x).$ 若满足 $\lim\limits_{n\to\infty} \int_a^b f_n(x) ~ \mathrm{d}x = \int_a^b f(x) ~ \mathrm{d}x,$ 那么 $f_n(x)$ 在 $D$ 上一致收敛到 $f(x).$
\item 设 $f$ 是定义在 $\mathbb{R}^n$ 上取值在 $\mathbb{R}^m$ 中的连续向量值函数, $E$ 是 $\mathbb{R}^n$ 中的非空有界开集, 那么 $E$ 在 $f$ 下的像集 $f(E)$ 必然是 $\mathbb{R}^m$ 中的开集.
\item 设数项级数 $\displaystyle \sum_{n=1}^{\infty} a_n, \sum_{n=1}^{\infty} b_n$ 满足 $|a_n| > |b_n|, ~ \forall n \in \mathbb{N}_+,$ 那么若 $\displaystyle \sum_{n=1}^{\infty} a_n$ 收敛, 则 $\displaystyle \sum_{n=1}^{\infty} b_n$ 必然也收敛.
\end{choices}
\end{question}

\begin{solution}
A 还需要进一步要求 $x_n \neq x,$ 排除孤立点的情况.

% C 一致收敛是积分与极限可交换的充分条件,但不是必要条件.

C 中的映射是所谓的开映射, 不一定连续; 连续的映射也不一定是开映射.

D 的反例可以取 $\displaystyle \sum_{n=1}^{\infty} a_n$ 为任何一个通项非零的条件收敛级数, 例如 $\displaystyle \sum_{n=1}^{\infty} (-1)^{n-1} \dfrac{1}{n},$ 再取 $b_n = \frac{1}{2}|a_n|,$ 那么 $|a_n| > |b_n|, ~ \forall n \in \mathbb{N}_+,$ 但 $\displaystyle \sum_{n=1}^{\infty} b_n = \frac{1}{2} \sum_{n=1}^{\infty} |a_n|$ 发散.

B 正确的原因: $\mathbb{R}^n$ 中的有界闭集是紧集, 紧集在连续映射下的像是紧集. $\mathbb{R}^m$ 中的紧集都是有界闭集.
\end{solution}

\begin{question}
设 $\{f_n(x)\}_{n\in\mathbb{N}}$ 是定义在 $[0, 1]$ 区间上的黎曼可积函数列, 并且有公共的界, 即存在正实数 $M,$ 使得 $|f_n(x)| \leqslant M$ 对任意的 $n \in \mathbb{N},$ 以及任意的 $x \in [0, 1]$ 都成立. 若 $\{f_n(x)\}_{n\in\mathbb{N}}$ 的极限函数存在 $\displaystyle f(x) = \lim_{n\to\infty} f_n(x),$ 那么以下说法\chemph{正确}的是 \paren[D]

\begin{choices}
\item 若积分值序列的极限 $\displaystyle \lim_{n\to\infty} \int_0^{+\infty} f_n(x) ~ \mathrm{d}x$ 也存在, 则极限函数 $f(x)$ 必然黎曼可积
\item 若积分值序列的极限 $\displaystyle \lim_{n\to\infty} \int_0^{+\infty} f_n(x) ~ \mathrm{d}x$ 也存在, 则 $\{f_n(x)\}_{n\in\mathbb{N}}$ 必然一致收敛到 $f(x)$
\item 若极限函数 $f(x)$ 黎曼可积, 则 $\{f_n(x)\}_{n\in\mathbb{N}}$ 必然一致收敛到 $f(x)$
\item 若极限函数 $f(x)$ 黎曼可积, 则积分值序列的极限 $\displaystyle \lim_{n\to\infty} \int_0^{+\infty} f_n(x) ~ \mathrm{d}x$ 也必存在, 并且有 $\displaystyle \lim_{n\to\infty} \int_0^{+\infty} f_n(x) ~ \mathrm{d}x = \int_0^{+\infty} f(x) ~ \mathrm{d}x$
\end{choices}
\end{question}

\begin{solution}
A 的反例: 令 $\mathbb{Q} \cap* [0, 1] = \{r_1, r_2, \dots\},$ 定义 $f_n(x) = \begin{cases}
1, & x \in \{r_1, r_2, \dots, r_n\},\\
0, & \text{其余情况}.
\end{cases}$ 那么 $\displaystyle \int_0^1 f_n(x) ~ \mathrm{d}x = 0,$ 其极限等于 $0,$ 但 $\{f_n(x)\}_{n\in\mathbb{N}}$ 的极限函数为狄利克雷函数, 不是黎曼可积的.

B 的反例同上. 容易看到对任意 $n \in \mathbb{N},$ 总存在 $x,$ 例如 $r_{n+1},$ 使得 $f_n(x) = 0,$ 而 $f(x) = 1,$ 所以 $\{f_n(x)\}_{n\in\mathbb{N}}$ 不是一致收敛到 $f(x)$ 的.

C 的反例: $f_n(x) = x^n,$ 极限函数为 $f(x) = \begin{cases}
0, & 0 \leqslant x < 1, \\
1, & x = 1.
\end{cases}$ $\{f_n(x)\}_{n\in\mathbb{N}}$ 不是一致收敛到 $f(x)$ 的.

D 是黎曼可积函数空间的有界收敛定理.
\end{solution}

\begin{question}
设幂级数 $\sum\limits_{n=0}^\infty a_n x^n$ 的收敛半径是$R,$ $0 < R < +\infty,$ 那么下面\chemph{正确}的论断是 \paren[C]

\begin{choices}
\item 极限 $\lim\limits_{n\to\infty} \nroot[n]{\lvert a_n \rvert}$ 必然存在且等于 $\dfrac{1}{R}$
\item 极限 $\lim\limits_{n\to\infty} \left\lvert \dfrac{a_{n+1}}{a_n} \right\rvert$ 必然存在且等于 $\dfrac{1}{R}$
\item 极限 $\lim\limits_{n\to\infty} \left\lvert \dfrac{a_{n+1}}{a_n} \right\rvert$ 可能不存在, 若极限存在则必等于 $\dfrac{1}{R}$
\item 以上说法都不对
\end{choices}
\end{question}

\begin{solution}
有不等式
\[A = \varliminf_{n\to\infty} \left\lvert \dfrac{a_{n+1}}{a_n} \right\rvert \leqslant \varliminf_{n\to\infty} \nroot[n]{\lvert a_n \rvert} \leqslant \varlimsup_{n\to\infty} \nroot[n]{\lvert a_n \rvert} \leqslant \varlimsup_{n\to\infty}  \left\lvert \dfrac{a_{n+1}}{a_n} \right\rvert,\]
而收敛半径等于 $1 / \varlimsup_{n\to\infty} \nroot[n]{\lvert a_n \rvert},$ 因此相关极限的存在性都不是必然的, 但是极限若存在的话, 必然都会等于 $1 / R.$
\end{solution}


\section{填空题:本题共 5 小题, 每小题 3 分, 共 15 分。}

\examsetup{
  question/index = 1
}

% \noindent\scoringbox

\begin{question}
柯西主值积分 $\displaystyle (\text{cpv}) \int_{-1}^1 \dfrac{2}{2x-1} \mathrm{d}x =$ \fillin[$-\ln 3$].
\end{question}

\begin{solution}
奇点在 $x = \frac{1}{2},$ 积分分为两部分:
\begin{align*}
\int_{-1}^{1/2 - \varepsilon} \dfrac{2}{2x-1} \mathrm{d}x & = \ln (1 - 2x) \bigg|_{-1}^{1/2 - \varepsilon} = \ln(2\varepsilon) - \ln 3, \\
\int_{1/2 + \varepsilon}^1 \dfrac{2}{2x-1} \mathrm{d}x & = \ln (2x - 1) \bigg|_{1/2 + \varepsilon}^1 = -\ln(2\varepsilon),
\end{align*}
相加, 取极限 $\varepsilon \to 0+,$ 得 $-\ln 3.$
\end{solution}

\begin{question}
设 $a > 0$ 为常数, 那么星形线 $\begin{cases}
x = a \cos^3 t, \\ y = a \sin^3 t
\end{cases}$ 的长度等于 \fillin[$6a$].
\end{question}

\begin{solution}
由对称性, 可只计算 $0 \leqslant t \leqslant \dfrac{\pi}{2}$ 对应的弧长, 整个曲线的长度等于 4 倍的此弧长. 接下来可以直接利用公式
\[\ell = \int_0^{\pi/2} \sqrt{(x_t')^2 + (y_t')^2} ~ \mathrm{d} t\]
进行计算.
\end{solution}

\begin{question}
积分 $\displaystyle \int_{-\pi}^{3\pi} \cos (2025 x) \cos(2024 x) ~ \mathrm{d} x =$ \fillin[$0$].
\end{question}

\begin{solution}
$\cos nx, n = 0, 1, 2, \dots; \sin m x, m = 1, 2, \dots$ 在任何一个长度为 $2\pi$ 的区间 $[a,b]$ 上构成一个关于 $\langle f, g \rangle = \int_{a}^{b} f(x)g (x) ~ \mathrm{d} x$ 正交函数系. 这里的积分区间长度 $4\pi$ 正好是 $2\pi$ 的 $2$ 倍.
\end{solution}

\begin{question}
幂级数 $\sum\limits_{n=1}^{\infty} \dfrac{(-2)^n \ln n}{n} (x - 1)^n$ 的收敛域为 \fillin[ ${\left( \frac{1}{2}, \frac{3}{2} \right]}$ ].
\end{question}

\begin{solution}
收敛半径 $R = 1 \left/ \varlimsup\limits_{n\to\infty} \nroot[n]{\left|\frac{(-2)^n \ln n}{n}\right|} \right. = \frac{1}{2}.$ 在端点 $x - 1 = \frac{1}{2}$ 处, 为 Leibniz 级数 $\sum\limits_{n=1}^{\infty} \dfrac{(-1)^n \ln n}{n}$ 收敛; 在端点 $x - 1 = -\frac{1}{2}$ 处为发散的正项级数 $\sum\limits_{n=1}^{\infty} \dfrac{\ln n}{n}.$
\end{solution}

\begin{question}
设 $\{a_n\}_{n\in\mathbb{N}}$ 为斐波那契数列, 即满足 $a_1=a_2=1,$ $a_{n+2} = a_{n+1}+a_n,$ 那么正项级数 $\displaystyle \sum_{n=1}^{\infty} \dfrac{a_n}{3^n} = $ \fillin[$\frac{3}{5}$].
\end{question}

\begin{solution}
容易看出 $a_n < 2^n,$ 那么可知 $\displaystyle \sum_{n=1}^{\infty} \dfrac{a_n}{3^n}$ 收敛, 设该值为 $A,$ 那么
\begin{align*}
3A & = \sum_{n=1}^{\infty} \dfrac{a_n}{3^{n-1}} = \sum_{n=1}^{\infty} \dfrac{a_{n+1}}{3^{n}} + a_1 \\
9A & = \sum_{n=1}^{\infty} \dfrac{a_n}{3^{n-2}} = \sum_{n=1}^{\infty} \dfrac{a_{n+2}}{3^{n}} + a_2 + 3a_1,
\end{align*}
于是有 $A + (3A - a_1) = 9A - a_2 - 3a_1,$ 解得 $A = \dfrac{3}{5}.$
\end{solution}


\section{计算题:本题共 2 小题, 共 20 分。本题应写出具体演算步骤。}

\examsetup{
  question/index = 1
}

\begin{question}[points = 10]
考虑二元函数 $f(x,y) = x\sin\dfrac{1}{y} + y\sin\dfrac{1}{x},$ 问二重极限 $\lim\limits_{(x,y) \to (0,0)} f(x,y)$ 以及二次极限 $\lim\limits_{x \to 0} \lim\limits_{y \to 0} f(x,y)$ 是否分别存在? 若存在, 求出相应的值; 若不存在, 说明原因.

% \noindent\scoringbox
\end{question}

\begin{solution}
二重极限 $\lim\limits_{(x,y) \to (0,0)} f(x,y)$ 存在. \score{2} 由于
\begin{equation*}
\left| x\sin\dfrac{1}{y} + y\sin\dfrac{1}{x} \right| \leqslant |x| \cdot \left| \sin\dfrac{1}{y} \right| + |y| \cdot \left| \sin\dfrac{1}{x} \right| \leqslant |x| + |y| \leqslant \sqrt{2(x^2 + y^2)} \score{2}
\end{equation*}
所以 $\lim\limits_{(x,y) \to (0,0)} f(x,y) = 0.$ \score{1}

二次极限 $\lim\limits_{x \to 0}\lim\limits_{y \to 0} f(x,y)$ 不存在. \score{2} 对固定的 $x \neq 0,$ 极限 $\lim\limits_{y \to 0} x\sin\dfrac{1}{y}$ 不存在, 极限 $\lim\limits_{y \to 0} y\sin\dfrac{1}{x} = 0,$ \score{2} 所以极限 $\lim\limits_{y \to 0} f(x,y) = \lim\limits_{y \to 0} \left( x\sin\dfrac{1}{y} + y\sin\dfrac{1}{x} \right)$ 不存在, 进而二次极限不存在. \score{1}
\end{solution}


\begin{question}[points = 10]
设 $n \in \mathbb{N}_+$ 为正整数, 请计算定积分 $\displaystyle I_n = \int_0^{\frac{\pi}{2}} \left(\dfrac{\sin (nx)}{\sin x}\right)^2 \mathrm{d}x.$ (提示: 先计算 $I_{n+1} - I_n.$)

% \noindent\scoringbox
\end{question}

\begin{solution}
容易算出 \score{2}
\begin{equation*}
\begin{aligned}
I_1 & = \int_0^{\frac{\pi}{2}} \mathrm{d}x = \dfrac{\pi}{2}, \\
I_2 & = \int_0^{\frac{\pi}{2}} \left(\dfrac{\sin (2x)}{\sin x}\right)^2 \mathrm{d}x = \int_0^{\frac{\pi}{2}} 4 \cos^2 x \mathrm{d}x = \int_0^{\frac{\pi}{2}} 2(1 - \sin(2x)) \mathrm{d}x = \pi.
\end{aligned}
\end{equation*}
对一般情况, 有
\begin{equation*}
\begin{aligned}
I_{n+1} - I_{n} & = \int_0^{\frac{\pi}{2}} \dfrac{\sin^2((n+1)x) - \sin^2(nx)}{\sin^2 x} \mathrm{d}x \\
& = \int_0^{\frac{\pi}{2}} \dfrac{\cos(2nx) - \cos(2(n+1)x)}{2\sin^2 x} \mathrm{d}x \\
& = \int_0^{\frac{\pi}{2}} \dfrac{\sin x (\sin(2n+1)x)}{\sin^2 x} \mathrm{d}x \\
& = \int_0^{\frac{\pi}{2}} \dfrac{\sin((2n+1)x)}{\sin x} \mathrm{d}x =: J_n.
\end{aligned} \score{4}
\end{equation*}
对于 $J_n$ 有
\begin{equation*}
\begin{aligned}
J_{n+1} - J_n & = \int_0^{\frac{\pi}{2}} \dfrac{\sin((2n+3)x) - \sin((2n+1)x)}{\sin x} \mathrm{d}x \\
& = \int_0^{\frac{\pi}{2}} \dfrac{2 \cos(2(n+1)x) \sin x}{\sin x} \mathrm{d}x \\
& = \int_0^{\frac{\pi}{2}} 2 \cos(2(n+1)x) ~ \mathrm{d}x = 0.
\end{aligned} \score{2}
\end{equation*}
于是 $\displaystyle J_n = J_1 = I_2 - I_1 = \dfrac{\pi}{2}.$ 即有 $I_{n+1} - I_n = J_n = \dfrac{\pi}{2},$ 所以
\begin{equation*}
I_n = I_1 + \dfrac{\pi}{2}(n - 1) = \dfrac{n\pi}{2}. \score{1}
\end{equation*}
\end{solution}


\section{解答题:本题共 5 小题, 共 50 分。解答应写出文字说明或者证明过程。\chemph{注意,若一道题分为多个小问,则该题前面小问的结论可以用于后面的小问,但反过来不行}。}

\examsetup{
  question/index = 1
}

\begin{question}[points = 8]
设 $\{ x_n \}_{n \in \mathbb{N}}$ 为一实数列, 请证明 $\varlimsup\limits_{n\to\infty} (-x_n) = - \varliminf\limits_{n\to\infty} x_n.$

% \noindent\scoringbox
\end{question}

\begin{solution}
% 若 $\varlimsup\limits_{n\to\infty} (-x_n) = +\infty,$ 则存在子列 $\{ x_{n_k} \}_{k \in \mathbb{N}},$ 使得 $\lim\limits_{k\to\infty} (-x_{n_k}) = +\infty,$ 那么 $\lim\limits_{k\to\infty} x_{n_k} = -\infty,$ 于是有 $\varliminf\limits_{n\to\infty} x_n = -\infty.$ 类似地可以从 $\varlimsup\limits_{n\to\infty} (-x_n) = -\infty$ 推出 $\lim\limits_{k\to\infty} x_{n_k} = +\infty.$

% 若 $\varlimsup\limits_{n\to\infty} (-x_n) = c \in \mathbb{R},$
记
\begin{align*}
E_- & = \{ -\infty \leqslant \xi \leqslant +\infty ~:~ \text{$\xi$ 为 $\{ -x_n \}_{n \in \mathbb{N}}$ 的极限点} \}, \\
E_+ & = \{ -\infty \leqslant \eta \leqslant +\infty ~:~ \text{$\eta$ 为 $\{ x_n \}_{n \in \mathbb{N}}$ 的极限点} \}
\end{align*}
那么 $\varlimsup\limits_{n\to\infty} (-x_n) = \max E_-.$ 这里, 若 $+\infty \in E_-,$ 则约定 $\max E_- = +\infty.$ 任取 $\xi \in E_-,$ 存在子列 $\{ -x_{n_k} \}_{k \in \mathbb{N}},$ 使得 $\lim\limits_{k\to\infty} (-x_{n_k}) = \xi,$ 那么 $\lim\limits_{k\to\infty} x_{n_k} = -\xi,$ 故 $-\xi$ 是 $\{ x_n \}_{n \in \mathbb{N}}$ 的极限点, 即 $-\xi \in E_+.$ 类似地可以从 $\eta \in E_+$ 推出 $-\eta \in E_-.$ 于是
\begin{align*}
\varlimsup\limits_{n\to\infty} (-x_n) & = \max E_- = \max \{ -\eta ~:~ \eta \in E_+\} = - \min \{ \eta ~:~ \eta \in E_+\} \\
& = - \min E_+ = - \varliminf\limits_{n\to\infty} x_n.
\end{align*}
\end{solution}


\begin{question}[points = 10]
设函数 $f(x)$ 在点 $x_0$ 处无穷次可微, 请问是否必然存在 $x_0$ 的某个邻域 $O(x_0, \rho) = (x_0 - \rho, x_0 + \rho),$ $\rho > 0,$ 使得 $f(x)$ 在 $O(x_0, \rho)$ 上可以展开成幂级数? 若是, 请给出证明; 若否, 请举反例并简要说明该反例不能展开成幂级数的原因.

% \noindent\scoringbox
\end{question}

\begin{solution}
不是.

反例可以举 (不限于) $f(x) = \begin{cases}
e^{-1/x^2}, & x \neq 0, \\ 0, & x = 0.
\end{cases}$

原因: {\color{red} 通过计算}可得 $f(x)$ 在 $0$ 点的各阶导数 $f^{(n)}(0) = 0,$ 那么若 $f(x)$ 在 $0$ 的某个邻域 $O(0, \rho)$ 有幂级数展开, 则该幂级数恒等于 $0,$ 但 $f(x)$ 在 $0$ 的任何邻域内都不恒等于 $0.$
\end{solution}


\begin{question}[points = 10]
设 $x = (x_1, \dots, x_n) \in \mathbb{R}^n,$ $E$ 是 $\mathbb{R}^n$ 中的非空点集, 定义点 $x$ 到集合 $E$ 的距离为
\[\rho(x, E) = \inf_{y \in E} ~ \lVert x - y \rVert,\]
其中 $\lVert x - y \rVert = \sqrt{(x_1 - y_1)^2 + \cdots + (x_n - y_n)^2}.$
\begin{enumerate}
\item 设 $F$ 为 $\mathbb{R}^n$ 中的非空闭集, 证明存在点 $y_0 \in F,$ 使得 $\rho(x, F) = \lVert x - y_0 \rVert.$
\item 设 $F_1, F_2$ 为 $\mathbb{R}^n$ 中两个不交的非空闭集, 请构造一个 $\mathbb{R}^n$ 上的\chemph{连续函数} $f(x),$ 同时满足以下两个条件
\begin{enumerate}
\item $0 \leqslant f(x) \leqslant 1, ~ \forall ~ x \in \mathbb{R}^n;$
\item $f(x)$ 在 $F_1$ 上取值恒等于 $0,$ 在 $F_2$ 上取值恒等于 $1.$
\end{enumerate}
% \item 设 $F_1, F_2, F_3$ 为 $\mathbb{R}^n$ 中三个互不相交的非空闭集, 请构造一个 $\mathbb{R}^n$ 上的\chemph{连续函数} $f(x),$ 同时满足以下两个条件
% \begin{enumerate}
% \item $0 \leqslant f(x) \leqslant 1, ~ \forall ~ x \in \mathbb{R}^n;$
% \item $f(x)$ 在 $F_1$ 上取值恒等于 $0,$ 在 $F_2$ 上取值恒等于 $\dfrac{1}{2025},$ 在 $F_3$ 上取值恒等于 $1.$
% \end{enumerate}
\end{enumerate}

% \noindent\scoringbox
\end{question}

\begin{solution}
\begin{enumerate}
\item 若 $x \in F,$ 则取 $y_0 = x$ 即可. 下设 $x \not\in F$.

取闭球 $\overline{B} = \overline{B}(x, \delta)$ 使得 $\overline{B} \cap* F \neq \emptyset.$ 这样的 $\delta > 0$ 总是可以取到的. 这是因为 $F$ 是非空集合, 可以任取其中一点 $y \in F,$ 并令 $\delta = \lVert x - y \rVert > 0.$ 容易看出
\begin{equation*}
\begin{gathered}
\rho(x, y) \geqslant \delta, ~~ \forall y \in F \setminus \overline{B}, \\
\rho(x, y) \leqslant \delta, ~~ \forall y \in F \cap* \overline{B}. \\
\end{gathered}
\end{equation*}
$\overline{B} \cap* F$ 是有界闭集, 从而是紧集, 关于 $x$ 的连续函数 $\rho(x, F)$ 在其上能取到最小值, 即存在 $y_0 \in F,$ 使得 $\rho(x, F) = \rho(x, F \cap* \overline{B}) = \lVert x - y_0 \rVert.$ \score{5}

证法二: 由下确界定义, 可以取到 $F$ 中点列 $\{y_n\},$ 使得 $\lim\limits_{n\to\infty} \rho(x, y_n) = \rho(x, F).$ 这可以说明 $\{y_n\}$ 是有界点列, 从而存在收敛子列, 而 $F$ 是闭集, 该收敛子列收敛到 $F$ 中的某点, $y_0$ 即取为该点.

\item 由于 $F_1, F_2$ 为 $\mathbb{R}^n$ 中两个不交的非空闭集, 那么 $\forall ~ x \in \mathbb{R}^n,$ $\rho(x, F_1), \rho(x, F_2)$ 不同时为 $0,$ 否则根据第 (1) 问, 存在 $y_1 \in F_1, y_2 \in F_2,$ 使得
\[0 = \lVert x - y_1 \rVert = \lVert x - y_2 \rVert,\]
那么会有 $x = y_1$ 以及 $x = y_2,$ 于是 $x \in F_1 \cap* F_2,$ 这与 $F_1, F_2$ 不交的已知条件矛盾.

定义 $\mathbb{R}^n$ 上的函数
\begin{equation*}
f(x) = \dfrac{\rho(x, F_1)}{\rho(x, F_1) + \rho(x, F_2)}, ~~ x \in \mathbb{R}^n \score{5}
\end{equation*}
由于 $\rho(x, F_1), \rho(x, F_2)$ 都是关于 $x$ 的连续非负函数, 且分母恒大于 $0,$ 因此$f(x)$ 是 $\mathbb{R}^n$ 上的连续函数. 由于
\[0 \leqslant \rho(x, F_1) \leqslant \rho(x, F_1) + \rho(x, F_2),\]
因此 $0 \leqslant f(x) \leqslant 1, ~ \forall ~ x \in \mathbb{R}^n.$

对于 $x \in F_1,$ 有 $\rho(x, F_1) = 0,$ 从而有
\[f(x) = \dfrac{\rho(x, F_1)}{\rho(x, F_1) + \rho(x, F_2)} = \dfrac{0}{0 + \rho(x, F_2)} = 0;\]
对于 $x \in F_2,$ 有 $\rho(x, F_2) = 0,$ 从而有
\[f(x) = \dfrac{\rho(x, F_1)}{\rho(x, F_1) + \rho(x, F_2)} = \dfrac{\rho(x, F_1)}{\rho(x, F_1) + 0} = 1.\]

% \item 由于 $F_2, F_3$ 为 $\mathbb{R}^n$ 中不交的非空闭集, 由第 (2) 问知, 存在 $\mathbb{R}^n$ 上的连续函数 $g(x),$ 满足
% \begin{align*}
% & 0 \leqslant g(x) \leqslant 1, ~ \forall ~ x \in \mathbb{R}^n; \\
% & \text{$g(x)$ 在 $F_2$ 上取值恒等于 $0,$ 在 $F_3$ 上取值恒等于 $1.$}
% \end{align*}
% 考虑 $F_1$ 与 $F_2 \cup* F_3,$ 由已知条件, 他们是 $\mathbb{R}^n$ 中两个不交的非空闭集, 那么再次利用第 (2) 问, 存在 $\mathbb{R}^n$ 上的连续函数 $h(x),$ 满足
% \begin{align*}
% & 0 \leqslant h(x) \leqslant 1, ~ \forall ~ x \in \mathbb{R}^n; \\
% & \text{$h(x)$ 在 $F_1$ 上取值恒等于 $0,$ 在 $F_2 \cup* F_3$ 上取值恒等于 $1.$}
% \end{align*}
% 定义 $\mathbb{R}^n$ 上的函数 $f(x)$ 如下
% \begin{align*}
% f(x) & = \dfrac{1}{2025} h(x) (1 + 2024 \cdot g(x)) \score{5} \\
% & = \dfrac{1}{2025} \cdot \dfrac{\rho(x, F_1)}{\rho(x, F_1) + \rho(x, F_2 \cup* F_3)} \cdot \left( 1 + \dfrac{2024 \cdot \rho(x, F_2)}{\rho(x, F_2) + \rho(x, F_3)} \right)
% \end{align*}
% 容易验证函数 $f(x)$ 满足题设条件.

% 注意, 这题不要忘了 $0 \leqslant f(x) \leqslant 1$ 的前提条件.

% $f(x)$ 其他可行的例子包括:
% \begin{equation*}
% f(x) = \dfrac{\rho(x, F_1\cup* F_2) + \rho(x, F_1\cup* F_3)}{\rho(x, F_1\cup* F_2) + 2025 \rho(x, F_1\cup* F_3) + \rho(x, F_2\cup* F_3)}
% \end{equation*}
\end{enumerate}
\end{solution}


\begin{question}[points = 10]
设 $a > 0, b > 0$ 为常数, $f(x)$ 为定义在 $[0, + \infty)$ 上的连续函数, 并且当 $x \to +\infty$ 时有极限 $\lim\limits_{x \to +\infty} f(x) = c_{\infty} \in \mathbb{R}.$ 记 $c_0 = f(0).$
% 请用含 $a, b, c_0, c_{\infty}$ 的表达式来表示反常积分 $\displaystyle \int_0^{+\infty} \dfrac{f(ax) - f(bx)}{x} \mathrm{d}x$ 的值, 并由此计算反常积分 $\displaystyle \int_0^{+\infty} \ln \dfrac{1+2e^{-ax}}{1+2e^{-bx}} \cdot \dfrac{\mathrm{d}x}{x}$ 的值.
\begin{enumerate}
\item 请证明: $\displaystyle \int_0^{+\infty} \dfrac{f(ax) - f(bx)}{x} \mathrm{d}x = (c_0 - c_{\infty}) \ln \dfrac{b}{a}.$
\item 计算反常积分 $\displaystyle \int_0^{+\infty} \ln \dfrac{1+2e^{-ax}}{1+2e^{-bx}} \cdot \dfrac{\mathrm{d}x}{x}$ 的值.
\end{enumerate}

% \noindent\scoringbox
\end{question}

\begin{solution}
\begin{enumerate}
\item 依定义, 反常积分
\begin{align*}
\int_0^{+\infty} \dfrac{f(ax) - f(bx)}{x} \mathrm{d}x & = \lim_{\substack{\delta \to 0+ \\ A \to +\infty}} \int_{\delta}^A \dfrac{f(ax) - f(bx)}{x} \mathrm{d}x \\
& = \lim_{\substack{\delta \to 0+ \\ A \to +\infty}} \left( \int_{\delta}^A \dfrac{f(ax)}{x} \mathrm{d}x - \int_{\delta}^A \dfrac{f(bx)}{x} \mathrm{d}x \right) \\
& = \lim_{\substack{\delta \to 0+ \\ A \to +\infty}} \left( \int_{a\delta}^{aA} \dfrac{f(t)}{t} \mathrm{d}t - \int_{b\delta}^{bA} \dfrac{f(t)}{t} \mathrm{d}t \right) \\
& = \lim_{\substack{\delta \to 0+ \\ A \to +\infty}} \left( \int_{a\delta}^{b\delta} \dfrac{f(t)}{t} \mathrm{d}t - \int_{aA}^{bA} \dfrac{f(t)}{t} \mathrm{d}t \right) \\
& = \lim_{\delta \to 0+} \int_{a\delta}^{b\delta} \dfrac{f(t)}{t} \mathrm{d}t - \lim_{A \to +\infty} \int_{aA}^{bA} \dfrac{f(t)}{t} \mathrm{d}t.
\end{align*}
由于 $\dfrac{1}{t}$ 在 $(0, +\infty)$ 上恒正, 且 $f(x)$ 连续, 由积分第一中值定理, 存在落在以 $a\delta, b\delta$ 为端点的闭区间中的某点 $\xi,$ 使得
\begin{equation*}
\int_{a\delta}^{b\delta} \dfrac{f(t)}{t} \mathrm{d}t = f(\xi) \int_{a\delta}^{b\delta} \dfrac{1}{t} \mathrm{d}t = f(\xi) \ln \dfrac{b}{a}.
\end{equation*}
类似地, 存在落在以 $aA, bA$ 为端点的闭区间中的某点 $\eta,$ 使得
\begin{equation*}
\int_{aA}^{bA} \dfrac{f(t)}{t} \mathrm{d}t = f(\eta) \int_{aA}^{bA} \dfrac{1}{t} \mathrm{d}t = f(\eta) \ln \dfrac{b}{a}.
\end{equation*}
当 $\delta \to 0+,$ 有 $\xi \to 0+,$ 因此由 $f(x)$ 的连续性有
\begin{equation*}
\lim_{\delta \to 0+} \int_{a\delta}^{b\delta} \dfrac{f(t)}{t} \mathrm{d}t = \lim_{\delta \to 0+} f(\xi) \ln \dfrac{b}{a} = f(0) \ln \dfrac{b}{a} = c_0 \ln \dfrac{b}{a}.
\end{equation*}
当 $A \to +\infty,$ 有 $\eta \to +\infty,$ 因此有
\begin{equation*}
\lim_{A \to +\infty} \int_{aA}^{bA} \dfrac{f(t)}{t} \mathrm{d}t = \lim_{A \to +\infty} f(\eta) \ln \dfrac{b}{a} = c_{\infty} \ln \dfrac{b}{a}.
\end{equation*}
综上有
\begin{equation*}
\int_0^{+\infty} \dfrac{f(ax) - f(bx)}{x} \mathrm{d}x = = \lim_{\delta \to 0+} \int_{a\delta}^{b\delta} \dfrac{f(t)}{t} \mathrm{d}t - \lim_{A \to +\infty} \int_{aA}^{bA} \dfrac{f(t)}{t} \mathrm{d}t = (c_0 - c_{\infty}) \ln \dfrac{b}{a}.
\end{equation*}

\item 对于反常积分 $\displaystyle \int_0^{+\infty} \ln \dfrac{1+2e^{-ax}}{1+2e^{-bx}} \cdot \dfrac{\mathrm{d}x}{x},$ 令 $f(x) = \ln(1 + 2e^{-x}),$ 由第 (1) 小问结论有
\begin{align*}
\int_0^{+\infty} \ln \dfrac{1+2e^{-ax}}{1+2e^{-bx}} \cdot \dfrac{\mathrm{d}x}{x} & = (f(0) - \lim_{x \to +\infty} f(x)) \cdot \ln\dfrac{b}{a} = (\ln 3 - \ln 1) \cdot \ln\dfrac{b}{a} \\
& = \ln 3 \cdot \ln\dfrac{b}{a}.
\end{align*}
\end{enumerate}
\end{solution}


\begin{question}[points = 12]
设 $\{ p_n \}_{n \in \mathbb{N}}, \{ q_n \}_{n \in \mathbb{N}}$ 是通项恒不为零的数列, 满足 $\lim\limits_{n\to\infty} \dfrac{q_n}{q_n - q_{n-1}} \cdot \dfrac{p_n - p_{n-1}}{p_n} = c,$ $c \in \mathbb{R}.$ 
\begin{enumerate}
\item 假设数列 $\{ q_n \}_{n \in \mathbb{N}}$ 单调递增且 $\lim\limits_{n \to \infty} q_n = +\infty,$ 请证明: 对任意数列 $\{ a_n \}_{n \in \mathbb{N}},$ 若极限 $\lim\limits_{n \to \infty} b_n = \lim\limits_{n \to \infty} \dfrac{1}{p_n} \sum\limits_{k=1}^n p_ka_k = B$ 存在, 那么极限 $\lim\limits_{n \to \infty} \dfrac{1}{q_n} \sum\limits_{k=1}^n q_ka_k$ 也必存在, 且等于 $Bc.$ (\chemph{提示}: 由 $\displaystyle b_n = \dfrac{1}{p_n} \sum\limits_{k=1}^n p_ka_k$ 反推 $a_n$ 的表达式, 并代入 $\displaystyle \dfrac{1}{q_n} \sum\limits_{k=1}^n q_ka_k$ 中进行分析.)
\item 设级数 $\sum\limits_{n=1}^\infty a_n$ 收敛, $\{ q_n \}_{n \in \mathbb{N}}$ 单调递增且 $\lim\limits_{n \to \infty} q_n = +\infty,$ 证明 $\lim\limits_{n \to \infty} \dfrac{1}{q_n} \sum\limits_{k=1}^n q_ka_k = 0.$
\item 设级数 $\sum\limits_{n=1}^\infty a_n$ 收敛, 并且 $\{ a_n \}_{n \in \mathbb{N}}$ 单调递减, 证明 $\lim\limits_{n \to \infty} na_n = 0.$
\item 设级数 $\sum\limits_{n=1}^\infty a_n$ 收敛, 并且 $\{ na_n \}_{n \in \mathbb{N}}$ 单调递减, 证明 $\lim\limits_{n \to \infty} na_n \sum\limits_{k=1}^n \dfrac{1}{k} = 0,$ 并由此进一步证明 $\lim\limits_{n \to \infty} (n\ln n)a_n = 0.$
\end{enumerate}

% \noindent\scoringbox
\end{question}

\begin{solution}
\begin{enumerate}
\item 约定 $b_0 = 0,$ 对 $\displaystyle b_n = \dfrac{1}{p_n} \sum\limits_{k=1}^n p_ka_k$ 变形可得
\begin{equation*}
a_n = \dfrac{b_np_n - b_{n-1}p_{n-1}}{p_n} = b_n - b_{n-1} + b_{n-1}\dfrac{p_n-p_{n-1}}{p_n},
\end{equation*}
进一步可得
\begin{equation*}
\dfrac{1}{q_n} \sum\limits_{k=1}^n q_ka_k = \dfrac{1}{q_n} \sum\limits_{k=1}^n q_k(b_k - b_{k-1}) + \dfrac{1}{q_n} \sum\limits_{k=1}^n q_k b_{k-1}\dfrac{p_k-p_{k-1}}{p_k}
\end{equation*}
对于第一项 $\displaystyle \dfrac{1}{q_n} \sum\limits_{k=1}^n q_k(b_k - b_{k-1}),$ (由 Abel 变换) 有
\begin{align*}
\dfrac{1}{q_n} \sum\limits_{k=1}^n q_k(b_k - b_{k-1}) & = \dfrac{1}{q_n} \left( \sum\limits_{k=1}^n q_kb_k - \sum\limits_{k=1}^n q_kb_{k-1} \right) = \dfrac{1}{q_n} \left( \sum\limits_{k=1}^n q_kb_k - \sum\limits_{k=1}^{n-1} q_{k+1}b_k \right) \\
& = \dfrac{1}{q_n} \left( q_nb_n + \sum\limits_{k=1}^{n-1} (q_k - q_{k+1})b_k \right) = b_n + \dfrac{1}{q_n} \sum\limits_{k=1}^{n-1} (q_k - q_{k+1})b_k,
\end{align*}
于是由 Stolz 定理有
\begin{align*}
\lim_{n \to \infty} \dfrac{1}{q_n} \sum\limits_{k=1}^{n-1} q_k(b_k - b_{k-1}) & = \lim_{n \to \infty} b_n + \lim_{n \to \infty} \dfrac{1}{q_n} \sum\limits_{k=1}^{n-1} (q_k - q_{k+1})b_k \\
& = B + \lim_{n \to \infty} \dfrac{(q_{n-1} - q_n)b_{n-1}}{q_n - q_{n-1}} \\
& = B - \lim_{n \to \infty} b_{n-1} = B - B \\
& = 0.
\end{align*}
对于第二项 $\displaystyle \dfrac{1}{q_n} \sum\limits_{k=1}^n q_k b_{k-1}\dfrac{p_k-p_{k-1}}{p_k}$ 利用 Stolz 定理有
\begin{align*}
\lim_{n \to \infty} \dfrac{1}{q_n} \sum\limits_{k=1}^n q_k b_{k-1}\dfrac{p_k-p_{k-1}}{p_k} & = \lim_{n \to \infty} \dfrac{q_n b_{n-1}\dfrac{p_n-p_{n-1}}{p_n}}{q_n-q_{n-1}} \\
& = \lim_{n \to \infty} b_{n-1} \dfrac{q_n(p_n-p_{n-1})}{p_n(q_n-q_{n-1})} \\
& = Bc.
\end{align*}
综合可得
\begin{equation*}
\lim_{n \to \infty} \dfrac{1}{q_n} \sum\limits_{k=1}^n q_ka_k = Bc.
\end{equation*}

\item 取 $p_n = 1$ 为常数列, 那么
\[\lim\limits_{n\to\infty} \dfrac{q_n}{q_n - q_{n-1}} \cdot \dfrac{p_n - p_{n-1}}{p_n} = \lim\limits_{n\to\infty} \dfrac{q_n}{q_n - q_{n-1}} \cdot \dfrac{0}{1} = 0,\]
并且
\[\lim\limits_{n \to \infty} \dfrac{1}{p_n} \sum\limits_{k=1}^n p_ka_k = \lim\limits_{n \to \infty} \sum\limits_{k=1}^n a_k \quad \text{收敛,}\]
于是由第 (1) 问的结论知
\begin{equation*}
\lim\limits_{n \to \infty} \dfrac{1}{q_n} \sum\limits_{k=1}^n q_ka_k = \left( \lim\limits_{n \to \infty} \sum\limits_{k=1}^n a_k \right) \cdot 0 = 0.
\end{equation*}

\item 由于级数 $\sum\limits_{n=1}^\infty a_n$ 收敛, 并且 $\{ a_n \}_{n \in \mathbb{N}}$ 单调递减, 容易证明 $a_n > 0$ 且 $\lim\limits_{n\to\infty} a_n = 0;$ 或者存在 $N \in \mathbb{N},$ 使得 $\forall ~ n > N,$ $a_n = 0.$ 后一种情况平凡, 于是以下假设 $a_n > 0.$ 那么可以取 $q_n = \dfrac{1}{a_n},$ 则 $\{ q_n \}_{n \in \mathbb{N}}$ 单调递增且 $\lim\limits_{n \to \infty} q_n = +\infty,$ 由第 (2) 问可知
\begin{equation*}
0 = \lim\limits_{n \to \infty} \dfrac{1}{q_n} \sum\limits_{k=1}^n q_ka_k = \lim\limits_{n \to \infty} a_n \sum\limits_{k=1}^n \dfrac{1}{a_k} \cdot a_k = \lim\limits_{n \to \infty} n a_n.
\end{equation*}

\item 首先, 必然有 $a_n > 0,$ 或者存在 $N \in \mathbb{N},$ 使得 $\forall ~ n > N,$ $a_n = 0;$ 否则, 若 $a_{n_0} < 0,$ 则由 $\{ na_n \}_{n \in \mathbb{N}}$ 单调递减知 $\forall ~ n > n_0,$ $a_n \leqslant \dfrac{n_0 a_{n_0}}{n},$ 从而 $\sum\limits_{n=1}^\infty a_n$ 必然发散. 存在 $N \in \mathbb{N},$ 使得 $\forall ~ n > N,$ $a_n = 0$ 的情况平凡, 以下都假设 $a_n > 0,$ 于是取 $q_n = \dfrac{1}{n a_n},$ 由第 (2) 问知
\begin{equation*}
0 = \lim\limits_{n \to \infty} \dfrac{1}{q_n} \sum\limits_{k=1}^n q_ka_k = \lim\limits_{n \to \infty} n a_n \sum\limits_{k=1}^n \dfrac{1}{k a_k} \cdot a_k = \lim\limits_{n \to \infty} n a_n \sum\limits_{k=1}^n \dfrac{1}{k}.
\end{equation*}
由于 $\lim\limits_{n \to \infty} \left( \sum\limits_{k=1}^n \dfrac{1}{k} - \ln n \right) = \gamma$ 极限存在, 所以
\begin{equation*}
\lim_{n \to \infty} (n\ln n)a_n = \lim_{n \to \infty} n a_n \sum\limits_{k=1}^n \left( \dfrac{1}{k} + \gamma \right) = \lim_{n \to \infty} n a_n \sum\limits_{k=1}^n \dfrac{1}{k} + \gamma \lim_{n \to \infty} n a_n = 0 + 0 = 0.
\end{equation*}
\end{enumerate}
\end{solution}

% \end{xiaosihao}

\end{document}
