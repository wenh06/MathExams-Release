\makeatletter
\@ifundefined{ifShowAnswer}{%
  \newif\ifShowAnswer
}{}
\makeatother

% \ShowAnswerfalse
% \ShowAnswertrue

\examsetup{
  page = {
    size            = a4paper,
    show-columnline = true,
    foot-content    = {第~;~页~~共~;页\quad 数学分析II\quad 中国农业大学制}
  },
  solution = {
    show-solution = show-stay,
    % blank-type = manual,
    blank-type = none,
    % blank-vsep = 120ex plus 1ex minus 1ex
    blank-vsep = 15cm
  },
  fillin = {
    no-answer-type = none,
    show-answer = true
  },
  % style/fullwidth-stop = catcode,
  style/fullwidth-stop = false,
  % sealline = {
  %   show        = true,
  %   scope       = mod-2,
  %   circle-show = false,
  %   line-type   = solid,
  %   odd-info-content = {
  %     {\heiti \zihao{4}姓名} {\underline{\hspace*{8em}}},
  %     {\heiti \zihao{4}准考证号} {\examsquare{9}},
  %     {\heiti \zihao{4}考场号} {\examsquare{2}},
  %     {\heiti \zihao{4}座位号} {\examsquare{2}},
  %   },
  %   odd-info-xshift = 12mm,
  %   text = {此卷只装订不密封},
  %   text-width = 0.98\textheight,
  %   text-format  = \zihao{-3}\sffamily,
  %   text-xshift = 20mm
  % },
  question/show-points = true,
  paren = {
    show-answer = true
  },
  square = {
    x-length = 1.8em,
    y-length = 1.6em
  },
  font = times
}

\ifShowAnswer
\examsetup{
  solution/show-solution = show-stay,
  fillin/show-answer = true,
  paren/show-answer = true,
  page/size = a3paper
}
\else
\examsetup{
  solution/show-solution = hide,
  fillin/show-answer = false,
  paren/show-answer = false,
  page/size = a4paper,
  page/show-head = true,
  page/head-content = {
  \fancyhead[LE]{\xiaosihao 学院:\rule[-0.45mm]{2.5cm}{0.15mm} \hspace{0.0cm} 班级:\rule[-0.45mm]{2.5cm}{0.15mm} \hspace{0.0cm} 学号:\rule[-0.45mm]{3.5cm}{0.15mm} \hspace{0.0cm} 姓名:\rule[-0.45mm]{2.5cm}{0.15mm}}
  }
}
\fi


\ifShowAnswer
% do nothing
\else
\AtEndPreamble{%
\geometry{
left=20mm,
right=20mm,
top=20mm,
bottom=20mm,
% includehead=true,
% includefoot=true,
% heightrounded,
% showframe,% <--- just for debugging
% verbose,% <--- just for debugging
headsep=8pt
}
}
\fi


\title{
\erhao
\simli
\ifUseImageTitle
{\includegraphics[height=0.85\baselineskip]{figures/logo_cau_name.png}}\\
\else
中国农业大学\\
\fi
2023\textasciitilde 2024学年春季学期\\
\textbf{%
% \uline{\hspace{1.5cm}数学分析II\hspace{1.5cm}}}
\dunderline[-1pt]{0.9pt}{\hspace{0.5cm}数学分析II\hspace{0.5cm}}}
\ifShowAnswer
课程考试试题解答
\else
课程考试试题
\fi
}

\begin{document}

\maketitle

\ifShowAnswer
% do nothing
\else
\vspace{-0.7cm}

{
\begin{table}[H]
\sihao
\centering
\begin{tabular}{|wc{2cm}|wc{2cm}|wc{2cm}|wc{2cm}|wc{2cm}|wc{2.5cm}|}
\hline
题号 & 一 & 二 & 三 & 四 & 总分 \\ \hline
分数 & & & & & \\[12pt] \hline
\end{tabular}
\end{table}
}

\vspace{-0.7cm}

\begin{center}
% \textbf{\larger 全卷满分 100 分。考试用时 100 分钟。}
{\sihao (本试卷共~4~道大题)}
\end{center}

\vspace{-0.6cm}
\begin{center}
\textbf{\sihao 考生诚信承诺}
\end{center}
\vspace{-0.4cm}
% {\sihao 本人承诺自觉遵守考试纪律,诚信应考,服从监考人员管理。\\
% 本人清楚学校考试考场规则,如有违纪行为,将按照学校违纪处分规定严肃处理。}
% 注意,这里不强行超过 linewidth 的话,第二行会自动断行
\noindent\begin{minipage}[t]{1.05\linewidth}
{\sihao 本人承诺自觉遵守考试纪律,诚信应考,服从监考人员管理。\\
本人清楚学校考试考场规则,如有违纪行为,将按照学校违纪处分规定严肃处理。}
\end{minipage}

\fi


\section{%
  选择题:本题共 5 小题, 每小题 3 分, 共 15 分。
  在每小题给出的四个选项中, 只有一项是符合题目要求的。
}

% \noindent\scoringbox

\begin{question}
设幂级数 $\sum\limits_{n=0}^\infty a_n x^n$ 满足下极限 $\varliminf_{n\to\infty} \left\lvert \dfrac{a_{n+1}}{a_n} \right\rvert = A,$ $0 < A < +\infty,$ 那么下面正确的论断是 \paren[D]

\begin{choices}
\item $\sum\limits_{n=0}^\infty a_n x^n$ 的收敛半径必定等于 $\dfrac{1}{A}.$
\item $\sum\limits_{n=0}^\infty a_n x^n$ 的收敛半径可能大于 $\dfrac{1}{A}.$
\item $\sum\limits_{n=0}^\infty a_n x^n$ 的收敛半径必定小于 $\dfrac{1}{A}.$
\item 以上说法都不对
\end{choices}
\end{question}

\begin{solution}
有不等式
\[A = \varliminf_{n\to\infty} \left\lvert \dfrac{a_{n+1}}{a_n} \right\rvert \leqslant \varliminf_{n\to\infty} \nroot[n]{\lvert a_n \rvert} \leqslant \varlimsup_{n\to\infty} \nroot[n]{\lvert a_n \rvert} \leqslant \varlimsup_{n\to\infty}  \left\lvert \dfrac{a_{n+1}}{a_n} \right\rvert,\]
而收敛半径等于 $1 / \varlimsup_{n\to\infty} \nroot[n]{\lvert a_n \rvert},$ 因此而收敛半径小于 $\dfrac{1}{A},$ 等于 $\dfrac{1}{A}$ 都是可能的.

等于 $\dfrac{1}{A}$ 的例子: $a_n = \dfrac{1}{n},$ 那么 $1 = \varliminf_{n\to\infty} \left\lvert \dfrac{a_{n+1}}{a_n} \right\rvert = \varlimsup_{n\to\infty} \nroot[n]{\lvert a_n \rvert}.$

小于 $\dfrac{1}{A}$ 的例子: $a_n = \dfrac{3^{n + (n \mod 2)}}{5^n},$ 那么 $\dfrac{1}{5} = \varliminf_{n\to\infty} \left\lvert \dfrac{a_{n+1}}{a_n} \right\rvert < \varlimsup_{n\to\infty} \nroot[n]{\lvert a_n \rvert} = \dfrac{3}{5}.$
\end{solution}

\begin{question}
设 $f: \mathbb{R}^n \to \mathbb{R}^m$ 是一个连续映射, 以下说法不正确的是 \paren[B]

\begin{choices}
\item 若 $K \subset* \mathbb{R}^n$ 是紧集, 则它的像集 $f(K)$ 必然也是 $\mathbb{R}^m$ 中的紧集
\item 若 $K \subset* \mathbb{R}^n$ 是闭集, 则它的像集 $f(K)$ 必然也是 $\mathbb{R}^m$ 中的闭集
\item 若 $E \subset* \mathbb{R}^m$ 是闭集, 则它的原像集 $f^{-1}(E)$ 必然也是 $\mathbb{R}^n$ 中的闭集
\item 若 $E \subset* \mathbb{R}^m$ 是开集, 则它的原像集 $f^{-1}(E)$ 必然也是 $\mathbb{R}^n$ 中的开集
\end{choices}
\end{question}

\begin{solution}
(无界) 闭集在连续映射下的像一般不是闭集. 例如 $f(x) = \dfrac{1}{1 + x^2},$ 闭集 $\mathbb{R}$ (或者取 $[0, +\infty)$) 的像为 $(0, 1],$ 不是 $\mathbb{R}$ 中闭集.
\end{solution}

\begin{question}
设二元函数 $f(x, y)$ 在点 $(x_0, y_0) \in \mathbb{R}^2$ 的某个去心邻域 $\mathring{O}((x_0, y_0), \delta)$ 内有定义. 那么关于 $f(x, y)$ 在点 $(x_0, y_0)$ 处的二重极限 $\lim\limits_{(x,y)\to(x_0,y_0)} f(x,y)$ 以及二次极限 $\lim\limits_{x \to x_0} \lim\limits_{y \to y_0} f(x,y),$ $\lim\limits_{y \to y_0} \lim\limits_{x \to x_0} f(x,y)$ 的存在性以及取值的情况, 下面哪一种情况是不可能的 \paren[D]

\begin{choices}
\item $\lim\limits_{(x,y)\to(x_0,y_0)} f(x,y) = \lim\limits_{x \to x_0} \lim\limits_{y \to y_0} f(x,y) = \lim\limits_{y \to y_0} \lim\limits_{x \to x_0} f(x,y) = 1$
\item $\lim\limits_{(x,y)\to(x_0,y_0)} f(x,y)$ 不存在, $\lim\limits_{x \to x_0} \lim\limits_{y \to y_0} f(x,y) = 2, \lim\limits_{y \to y_0} \lim\limits_{x \to x_0} f(x,y) = 1$
\item $\lim\limits_{(x,y)\to(x_0,y_0)} f(x,y)$ 不存在, $\lim\limits_{x \to x_0} \lim\limits_{y \to y_0} f(x,y)$ 也不存在, $\lim\limits_{y \to y_0} \lim\limits_{x \to x_0} f(x,y) = 2$
% \item $\lim\limits_{(x,y)\to(x_0,y_0)} f(x,y)$ 不存在, $\lim\limits_{x \to x_0} \lim\limits_{y \to y_0} f(x,y) = 1, \lim\limits_{y \to y_0} \lim\limits_{x \to x_0} f(x,y) = 1$
\item $\lim\limits_{(x,y)\to(x_0,y_0)} f(x,y) = 1,$ $\lim\limits_{x \to x_0} \lim\limits_{y \to y_0} f(x,y) = 2,$ $\lim\limits_{y \to y_0} \lim\limits_{x \to x_0} f(x,y)$ 不存在.
\end{choices}
\end{question}

\begin{solution}
累次极限与重极限所有可能的取值情况如下

\begin{table}[H]
\centering
\input{tables/multi-limits}
\end{table}
\end{solution}

% \begin{question}
%   设 $f(x) = \begin{cases}
%       x^2 \sin \dfrac{1}{x} \cos \dfrac{1}{y}, & x \neq 0 ~ \text{且} ~ y \neq 0, \\
%       0, & x = 0 ~ \text{或} ~ y = 0.
%   \end{cases}$ 以下论断正确的是 \paren[]

%   \begin{choices}
%     \item $\lim\limits_{(x, y) \to (0, 0)} f(x, y)$ 存在; $\lim\limits_{x \to 0} \lim\limits_{y \to 0} f(x, y)$ 与 $\lim\limits_{y \to 0} \lim\limits_{x \to 0} f(x, y)$ 都不存在
%     \item $\lim\limits_{(x, y) \to (0, 0)} f(x, y)$ 与 $\lim\limits_{y \to 0} \lim\limits_{x \to 0} f(x, y)$ 都存在; $\lim\limits_{x \to 0} \lim\limits_{y \to 0} f(x, y)$ 不存在
%     \item $\lim\limits_{x \to 0} \lim\limits_{y \to 0} f(x, y)$ 与 $\lim\limits_{y \to 0} \lim\limits_{x \to 0} f(x, y)$ 都存在; $\lim\limits_{(x, y) \to (0, 0)} f(x, y)$ 不存在
%     \item $\lim\limits_{(x, y) \to (0, 0)} f(x, y),$ $\lim\limits_{x \to 0} \lim\limits_{y \to 0} f(x, y)$ 与 $\lim\limits_{y \to 0} \lim\limits_{x \to 0} f(x, y)$ 都存在
%   \end{choices}
% \end{question}

\begin{question}
设 $f$ 是定义在闭区间 $[a, b]$ 上的函数. 以下关于函数 $f$ 说法正确的是 \paren[A]

\begin{choices}
\item 若 $f$ 单调, 则 $f$ 必然 Riemann 可积
\item 若 $f$ 有界, 则 $f$ 必然 Riemann 可积
\item 若 $\int_a^b \lvert f(x) \rvert \mathrm{d} x = 0,$ 则 $f$ 在闭区间 $[a, b]$ 上恒等于 $0$
\item 若 $f$ 是 $[a, b]$ 上 Riemann 可积函数, 且值域包含于闭区间 $[A, B],$ $g$ 为定义在 $[A, B]$ 上的另一个 Riemann 可积函数, 则它们的复合 $g\circ f$ 也必然是 Riemann 可积的.
\end{choices}
\end{question}

\begin{solution}
注意, 定义在闭区间 $[a, b]$ 上的函数, 其单调性蕴含了有界性.

B. 的反例: $[0, 1]$ 上的 Dirichlet 函数

C. 的反例: $f(x) = \begin{cases}
1, & x = 1, \\
0, & x \in [0, 1)
\end{cases}$

D. 的反例: $f(x)$ 为 $[0, 1]$ 上的 Riemann 函数, $g(x) = \begin{cases}
1, & x \in (0, 1], \\
0, & x = 0
\end{cases},$ 则 $f, g$ 都 Riemann 可积, 但 $g\circ f$ 是 $[0, 1]$ 上的 Dirichlet 函数, 不是 Riemann 可积的.
\end{solution}

\begin{question}
设正项级数 $\sum\limits_{n=1}^{\infty} a_n$ 发散, $S_n = \sum\limits_{k=1}^{n} a_k,$ 满足 $\lim\limits_{n\to\infty} \dfrac{a_n}{S_n} = 0,$ 那么幂级数 $\sum\limits_{n=1}^{\infty} a_n x^n$ 的收敛半径为 \paren[C]

\begin{choices}
\item $+\infty$
\item $0$
\item $1$
\item 某个小于 $1$ 的正实数
\end{choices}
\end{question}

\begin{solution}
由 $\lim\limits_{n\to\infty} \dfrac{a_n}{S_n} = 0$ 可推出 $\lim\limits_{n\to\infty} \dfrac{S_{n+1}}{S_n} = 1,$ 由 d'Alembert 定理知 $\sum\limits_{n=1}^{\infty} S_n x^n$ 的收敛半径为 $1.$ 由于 $\sum\limits_{n=1}^{\infty} a_n$ 是正项级数, 所以 $a_n \leqslant S_n,$ 故 $\sum\limits_{n=1}^{\infty} s_n x^n$ 的收敛半径 $\geqslant 1.$

另一方面, $\sum\limits_{n=1}^{\infty} s_n x^n$ 在 $x = 1$ 处发散, 故其收敛半径 $\leqslant 1.$ 综合知, $\sum\limits_{n=1}^{\infty} s_n x^n$ 的收敛半径等于 $1.$
\end{solution}


\section{填空题:本题共 5 小题, 每小题 3 分, 共 15 分。}

\examsetup{
  question/index = 1
}

% \noindent\scoringbox

\begin{question}
定积分 $\int_{-3\pi/2}^{\pi/2} \sin(2x) \sin(5x) \mathrm{d} x = $ \fillin[0].
\end{question}

\begin{solution}
$\cos nx, n = 0, 1, 2, \dots; \sin m x, m = 1, 2, \dots$ 在任何一个长度为 $2\pi$ 的区间 $[a,b]$ 上构成一个关于 $\langle f, g \rangle = \int_{a}^{b} f(x)g (x) ~ \mathrm{d} x$ 正交函数系. 这题里 $[a, b] = [-3\pi/2, \pi/2].$
\end{solution}

\begin{question}
设函数 $S(x) = \int_0^x \lvert \sin t \rvert \mathrm{d} t,$ 则 $\lim\limits_{x\to+\infty} \dfrac{S(x)}{x} =$ \fillin[$\dfrac{2}{\pi}$].
\end{question}

\begin{solution}
对任意非负整数 $k,$ $\int_{k\pi}^{(k+1)\pi} \lvert \sin t \rvert \mathrm{d} t = 2.$ 令 $n = \left[ \dfrac{x}{\pi} \right],$ 那么
\[2n = \int_0^{n\pi} \lvert \sin t \rvert \mathrm{d} t \leqslant \int_0^x \lvert \sin t \rvert \mathrm{d} t \leqslant \int_0^{(n+1)\pi} \lvert \sin t \rvert \mathrm{d} t = 2(n+1).\]
那么在区间 $[n\pi, (n+1)\pi]$ 上有
\[\dfrac{2n}{(n+1)\pi} \leqslant \dfrac{\int_0^x \lvert \sin t \rvert \mathrm{d} t}{x} \leqslant \dfrac{2(n+1)}{n\pi}.\]
由夹逼准则知
\[\lim\limits_{x\to+\infty} \dfrac{S(x)}{x} = \lim\limits_{x\to+\infty} \dfrac{\int_0^x \lvert \sin t \rvert \mathrm{d} t}{x} = \dfrac{2}{\pi}.\]
\end{solution}

\begin{question}
若反常积分 $\int_1^{+\infty} \dfrac{\ln x}{x^p} \mathrm{d} x$ 收敛, 则实数 $p$ 可以取值的范围为 \fillin[$(1, +\infty)$ (或写 $p > 1$)].
\end{question}

\begin{solution}
任取 $1 < q < p,$ 有 $\lim\limits_{x\to+\infty} \left. \dfrac{\ln x}{x^p} \right/ \dfrac{1}{x^q} = \lim\limits_{x\to+\infty} \dfrac{\ln x}{x^{p-q}} = 0,$ 根据非负函数反常积分敛散性的比较判别法的极限形式, 由 $\int_1^{+\infty} \dfrac{1}{x^q} \mathrm{d} x$ 收敛知 $\int_1^{+\infty} \dfrac{\ln x}{x^p} \mathrm{d} x$ 收敛.
\end{solution}

\begin{question}
若幂级数 $\sum\limits_{n=0}^\infty a_n x^{2n+1}$ 的收敛半径是 $2$, 那么幂级数 $\sum\limits_{n=0}^\infty a_n x^n$ 的收敛半径是 \fillin[$4$].
\end{question}

\begin{solution}
由 Cauchy-Hadamard 定理, $2 = \varlimsup_{n\to\infty}\nroot[2n+1]{\lvert a_n
\rvert} = \left( \varlimsup_{n\to\infty}\nroot[n]{\lvert a_n \rvert} \right)^{1/2}$
\end{solution}

\begin{question}
二重极限 $\lim\limits_{(x,y) \to (0,0)} \dfrac{\sin(x^3 + y^5)}{x^2 + y^2} =$ \fillin[$0$].
\end{question}

\begin{solution}
令 $x = r\cos\theta, y = r\sin\theta,$ $r \geqslant 0, \theta \in [0, 2\pi).$ 不妨设 $r < 1,$ 那么有
\[\left\lvert \dfrac{\sin(x^3 + y^5)}{x^2 + y^2} \right\rvert \leqslant \left\lvert \dfrac{x^3 + y^5}{x^2 + y^2} \right\rvert = r \cdot \left\lvert r^2\sin^5\theta + \cos^3\theta \right\rvert \leqslant r \cdot \left( \left\lvert r^2\sin^5\theta \right\rvert + \left\lvert \cos^3\theta \right\rvert \right) \leqslant 2 r.\]
于是, 对任意 $\varepsilon > 0,$ 取 $\delta = \varepsilon / 2 > 0,$ 那么对任意满足 $\lVert (x, y) \rVert = \sqrt{x^2 + y^2} = r < \delta,$ 有
\[\left\lvert \dfrac{\sin(x^3 + y^5)}{x^2 + y^2} \right\rvert \leqslant 2r < 2\delta = \varepsilon,\]
所以二重极限 $\lim\limits_{(x,y) \to (0,0)} \dfrac{\sin(x^3 + y^5)}{x^2 + y^2} = 0.$
\end{solution}


\section{计算题:本题共 2 小题, 共 20 分。本题应写出具体演算步骤。}

\examsetup{
  question/index = 1
}

\begin{question}[points = 10]
求 Archimedes 螺线 $r(\theta) = a\theta, a > 0,$ 第一圈 (对应 $\theta \in [0, 2\pi]$) 的弧长.

% \noindent\scoringbox
\end{question}

\begin{solution}
由于 $r'(\theta) = a,$ 代入极坐标下的弧长公式有
\begin{align*}
\ell & = \int_{0}^{2\pi} \sqrt{\left(r(\theta)\right)^2 + \left(r'(\theta)\right)^2} \mathrm{d} \theta \\
& = \int_{0}^{2\pi} \sqrt{\left(a \theta\right)^2 + a^2} \mathrm{d} \theta = a \int_{0}^{2\pi} \sqrt{\theta^2 + 1} \mathrm{d} \theta \\
& = \left. \dfrac{a}{2} \left( x \sqrt{x^2 + 1} + \ln \left( x + \sqrt{x^2 + 1} \right) \right) \right|_{0}^{2\pi} \\
& = \dfrac{a}{2} \left( 2\pi \sqrt{4 \pi^2 + 1} + \ln \left( 2\pi + \sqrt{4 \pi^2 + 1} \right) \right)
\end{align*}
\end{solution}

\begin{question}[points = 10]
计算函数 $f(x) = x\cot x$ 在 $x = 0$ 附近直到 $x^4$ 的幂级数展开.

% \noindent\scoringbox
\end{question}

\begin{solution}
首先 $\lim\limits_{x \to 0} x\cot x = 1,$ 所以 函数 $f(x) = x\cot x$ 在 $x = 0$ 处有定义.

由于 $x\cot x$ 是偶函数, 可以用待定系数法令$x\cot x = c_0 + c_2 x^2 + c_4 x^4 + \cdots,$ 有
\[x\cos x = x\left( 1 - \dfrac{x^2}{2!} + \dfrac{x^4}{4!} + \cdots \right) = x\cot x \cdot \sin x = \left( c_0 + c_2 x^2 + c_4 x^4 + \cdots \right) \cdot \left(x - \dfrac{x^3}{3!} + \dfrac{x^5}{5!} + \cdots\right),\]
令对应系数相等, 得方程组
\[\begin{cases}
1 = c_0 \\
-\dfrac{1}{2} = c_2 - \dfrac{1}{6} \cdot c_0 \\
\dfrac{1}{24} = c_4 - \dfrac{1}{6} \cdot c_2 + \dfrac{1}{120} \cdot c_0
\end{cases}\]
解得 $c_0 = 1, c_3 = - \dfrac{1}{3}, c_4 = - \dfrac{1}{45}.$ 于是幂级数展开为
\[x\cot x = 1 - \dfrac{x^2}{3} - \dfrac{x^4}{45} + \cdots\]
\end{solution}

\section{解答题:本题共 5 小题, 共 50 分。解答应写出文字说明或者证明过程。\chemph{注意,若一道题分为多个小问,则该题前面小问的结论可以用于后面的小问,但反过来不行}。}

\examsetup{
  question/index = 1
}

\begin{question}[points = 8]
记 $\operatorname{M}_n(\mathbb{R})$ 为 $n$ 阶实方阵全体构成的集合, 通过如下的一一映射
\[\begin{pmatrix}
a_{11} & \cdots & a_{1n} \\
a_{21} & \cdots & a_{2n} \\
\vdots & & \vdots \\
a_{n1} & \cdots & a_{nn}
\end{pmatrix} \mapsto (a_{11}, \dots, a_{1n}, a_{21}, \dots, a_{2n}, \dots, a_{nn})\]
可以将 $\operatorname{M}_n(\mathbb{R})$ 视作 $n^2$ 维Euclid空间 $\mathbb{R}^{n^2}.$ 记 $\operatorname{M}_n(\mathbb{R})$ 中所有可逆方阵构成的集合为 $\operatorname{GL}_n(\mathbb{R}),$ 即
\[\operatorname{GL}_n(\mathbb{R}) = \left\{ A \in \operatorname{M}_n(\mathbb{R}): ~ \det A \neq 0 \right\}.\]
证明 $\operatorname{GL}_n(\mathbb{R})$ 在 $\operatorname{M}_n(\mathbb{R})$ 中不是道路连通的.

% \noindent\scoringbox
\end{question}

\begin{solution}
令矩阵 $A_1 = \operatorname{diag}(1, 1, \dots, 1)$ 为 $n$ 阶单位阵, 其行列式等于 $1;$ 令矩阵 $A_2 = \operatorname{diag}(-1, 1, \dots, 1)$ 为将矩阵 $A_1$ 的第 $1$ 行第 $1$ 列元素改为 $-1$ 所得矩阵, 其行列式等于 $-1.$ 假设 $\operatorname{GL}_n(\mathbb{R})$ 是道路连通的, 那么存在一条道路
\[\gamma: [0, 1] \longrightarrow \operatorname{GL}_n(\mathbb{R}),\]
使得 $\gamma(0) = A_1, \gamma(1) = A_2.$
由于行列式 $\det$ 是关于方阵元素的多项式, 因此是连续的. 于是
\[\det\circ\gamma: [0, 1] \longrightarrow \mathbb{R}\]
是连续的一元函数, 并且有 $\det(\gamma(0)) = \det A_1 = 1, \det(\gamma(1)) = \det A_2 = -1.$ 于是根据闭区间上连续函数的零点存在定理 (或者中间值定理), 存在 $\xi \in [0, 1],$ 使得 $0 = \det(\gamma(\xi)).$ 记 $B = \gamma(\xi),$ 那么上式表明 $\det B = 0,$ 从而有 $B \not\in \operatorname{GL}_n(\mathbb{R}),$ 这与 $\gamma$ 是 $\operatorname{GL}_n(\mathbb{R})$ 中的一条道路矛盾.
\end{solution}

\begin{question}[points = 10]
求证函数 $f(x) = \begin{cases}
e^{-1/x^2}, & x \neq 0, \\
0, & x = 0
\end{cases}$ 在任何形如 $(-a, a), ~ a > 0,$ 的区间上, 都不能表示为某个在 $(-a, a)$ 上收敛的幂级数的和函数.

% \noindent\scoringbox
\end{question}

\begin{solution}
容易算得当 $x \neq 0$ 时, $f(x)$ 的各阶导数为
\begin{align*}
f'(x) & = \dfrac{2}{x^3} e^{-1/x^2}, \\
f''(x) & = \left( \dfrac{4}{x^6} - \dfrac{6}{x^4} \right) e^{-1/x^2}, \\
& \vdots \\
f^{(k)}(x) & = P_{3k} \left( \dfrac{1}{x} \right) e^{-1/x^2}, \\
& \vdots
\end{align*}
其中 $P_{3k} \left( \dfrac{1}{x} \right)$ 是关于 $\dfrac{1}{x}$ 的 $3k$ 次多项式. 在 $x = 0$ 处的各阶导数可依次用定义求:
\begin{align*}
f'(0) & = \lim\limits_{x \to 0} \dfrac{f(x) - f(0)}{x} = \lim\limits_{x \to 0} \dfrac{e^{-1/x^2}}{x} = 0, \\
f''(0) & = \lim\limits_{x \to 0} \dfrac{f'(x) - f'(0)}{x} = \lim\limits_{x \to 0} \dfrac{\dfrac{2}{x^3} e^{-1/x^2}}{x} = 0, \\
& \vdots \\
f^{(k+2)}(0) & = \lim\limits_{x \to 0} \dfrac{f^{(k)}(x) - f^{(k)}(0)}{x} = \lim\limits_{x \to 0} \dfrac{P_{3k} \left( \dfrac{1}{x} \right) e^{-1/x^2}}{x} = 0, \\
& \vdots
\end{align*}
假设 $f(x)$ 等于某个在 $(-a, a)$ 上收敛的幂级数的和函数 $f(x) = \sum\limits_{k=0}^\infty a_k x^k$, 那么根据幂级数的逐项可导性有
\[a_k \cdot k! = f^{(k)}(0) = 0,\]
那么有 $a_k = 0$ 对所有的 $k = 0, 1, \dots$ 都成立, 相应幂级数的和函数显然等于常值函数 $0,$ 不等于 $f(x),$ 矛盾. 所以 $f(x)$ 在任何形如 $(-a, a), ~ a > 0,$ 的区间上, 都不能表示为某个在 $(-a, a)$ 上收敛的幂级数的和函数.
\end{solution}

\begin{question}[points = 10]
设函数项级数 (称为Dirichlet级数) $\sum\limits_{n=1}^\infty \dfrac{a_n}{n^x}$ 在 $x = x_0 \in \mathbb{R}$ 处收敛.

\begin{enumerate}
\item 证明 $\sum\limits_{n=1}^\infty \dfrac{a_n}{n^x}$ 在 $x \in [x_0, +\infty)$ 上一致收敛.
\item 任取 $x > x_0 + 1,$ 证明 $\sum\limits_{n=1}^\infty \dfrac{a_n}{n^x}$ 绝对收敛.
% \item 请问使得 $\sum\limits_{n=1}^\infty \dfrac{a_n}{n^x}$ 绝对收敛的下界 $x_0 + 1$ 还能改进吗? 也就是说, 是否存在 $0 < s < 1,$ 使得 $\sum\limits_{n=1}^\infty \dfrac{a_n}{n^x}$ 也必然绝对收敛? 若存在, 请写一个这样的 $s,$ 满足对任意 $x > x_0 + s,$ $\sum\limits_{n=1}^\infty \dfrac{a_n}{n^x}$ 必然绝对收敛; 若不存在, 请说明原因.
\end{enumerate}

% \noindent\scoringbox
\end{question}

\begin{solution}
\begin{enumerate}
% \item 对任意取定的 $x \geqslant x_0,$ (作为数项级数通项的) $\tau_n(x) := \dfrac{1}{n^{x-x_0}}$ 关于 $n$ 单调非增, 且以 $1$ 为界, 又有数项级数 $\sum\limits_{n=1}^\infty \dfrac{a_n}{n^{x_0}}$ 收敛, 于是根据数项级数的 Abel 判别法,
% \[\sum\limits_{n=1}^\infty \dfrac{a_n}{n^{x_0}} \cdot \tau_n(x) = \sum\limits_{n=1}^\infty \dfrac{a_n}{n^{x_0}} \cdot \dfrac{1}{n^{x-x_0}} = \sum\limits_{n=1}^\infty \dfrac{a_n}{n^{x}}\]
% 收敛. 于是 $\sum\limits_{n=1}^\infty \dfrac{a_n}{n^x}$ 的收敛域包含 $[x_0, +\infty).$
% \item 任取 $x_1 > x_2 \geqslant x_0,$ 记 $\delta = x_1 - x_2 > 0.$ 有
% \[\dfrac{a_n}{n^{x_2}} - \dfrac{a_n}{n^{x_1}} = \dfrac{a_n}{n^{x_2}} \left( 1 - \dfrac{1}{n^{x_1-x_2}} \right) = \dfrac{a_n}{n^{x_2}} \left( 1 - \dfrac{1}{n^{\delta}} \right) = \dfrac{a_n}{n^{x_0}} \left( 1 - \dfrac{1}{n^{\delta}} \right) \dfrac{1}{n^{x_2-x_0}}.\]
% 由于数项级数 $b_n := 1 - \dfrac{1}{n^{\delta}}$ 关于 $n$ 单调非减, 满足 $0 < b_n < 1,$ 又有数项级数 $\sum\limits_{n=1}^\infty \dfrac{a_n}{n^{x_0}}$ 收敛, 于是根据数项级数的 Abel 判别法, 数项级数
% \[\sum\limits_{n=1}^\infty \dfrac{a_n}{n^{x_0}} \left( 1 - \dfrac{1}{n^{\delta}} \right)\]
% 收敛, 那么 $\forall \varepsilon > 0,$ 存在正整数 $N$ (这个 $N$ 只与 $\varepsilon$ 以及 $x_0$ 有关, 与 $x_1, x_2$ 无关), 使得对任意 $n_2 > n_1 > N$ 有
% \[\sum\limits_{n=n_1+1}^{n_2} \dfrac{a_n}{n^{x_0}} \left( 1 - \dfrac{1}{n^{\delta}} \right)\]
\item 记 $\tau_n(x) := \dfrac{1}{n^{x-x_0}},$ 那么 $\tau_n(x)$ 对每个固定的 $x \in [x_0, +\infty)$ 关于 $n$ 单调非增, 且一致地以 $1$ 为界:
\[\lvert \tau_n(x) \rvert \leqslant 1, ~~ \forall x \in [x_0, +\infty), n \in \mathbb{N}.\]
又由于数项级数 $\sum\limits_{n=1}^\infty \dfrac{a_n}{n^{x_0}}$ 收敛, 把它视作关于 $x$ 的函数项级数, 则它关于 $x$ 一致收敛. 于是根据函数项级数的 Abel 判别法,
\[\sum\limits_{n=1}^\infty \dfrac{a_n}{n^{x_0}} \cdot \tau_n(x) = \sum\limits_{n=1}^\infty \dfrac{a_n}{n^{x_0}} \cdot \dfrac{1}{n^{x-x_0}} = \sum\limits_{n=1}^\infty \dfrac{a_n}{n^{x}}\]
在 $[x_0, +\infty)$ 上一致收敛.
\item 由于数项级数 $\sum\limits_{n=1}^\infty \dfrac{a_n}{n^{x_0}}$ 收敛, 因此其通项$\dfrac{a_n}{n^{x_0}} \to 0 ~ (n\to\infty).$ 于是, 存在正整数 $N,$ 使得对任意 $n > N$ 有
\[\left\lvert \dfrac{a_n}{n^{x_0}} \right\rvert = \left\lvert \dfrac{a_n}{n^{x_0}} - 0 \right\rvert < 1.\]
记 $s = x - x_0 > 1,$ 那么对任意 $n > N$ 有
\[\left\lvert \dfrac{a_n}{n^x} \right\rvert = \left\lvert \dfrac{a_n}{n^{x_0}} \right\rvert \cdot \dfrac{1}{n^s} < \dfrac{1}{n^s}.\]
$\sum\limits_{n=1}^\infty \dfrac{1}{n^s}$ 是收敛的正项级数, 因此由正项级数的比较判别法知 $\sum\limits_{n=1}^\infty \left\lvert \dfrac{a_n}{n^{x}} \right\rvert$ 也收敛.
\end{enumerate}
\end{solution}

\begin{question}[points = 10]
叙述并证明 $n$ 维 Euclid 空间 $\mathbb{R}^n$ 中的 Cantor 闭区域套定理.

% \noindent\scoringbox
\end{question}

\begin{solution}
Cantor 闭区域套定理: 设
\[D_1 \supset* D_2 \supset* \cdots \supset* D_k \supset* D_{k+1} \supset* \cdots\]
为 $\mathbb{R}^n$ 中的闭集 (闭区域) 套, 且满足 $\lim\limits_{k \to \infty} \operatorname{diam} D_k = 0,$ 其中
\[\operatorname{diam} D_k = \sup\limits_{x_1, x_2 \in D_k} \lVert x_1 - x_2 \rVert,\]
那么存在唯一一点 $x \in \bigcap\limits_{k=1}^\infty D_k.$

Cantor 闭区域套定理的证明: 首先证明 $x$ 的存在性, 即 $\bigcap\limits_{k=1}^{\infty} E_k \neq \emptyset.$

由于每个 $E_k$ 都是非空闭集, 于是可以取到点列 $x_k \in E_k.$ 由于对任意 $p \geqslant 1,$ 有 $E_{k+p} \subset* E_k,$ 于是 $x_{k+p} \in E_{k+p} \subset* E_k,$ 从而有 $d(x_k, x_{k+p}) \leqslant \operatorname{diam} E_k,$ 这表明 $\{ x_k \}_{k\in\mathbb{N}}$ 是 $\mathbb{R}^n$ 中的一个 Cauchy 列, 从而存在 $x \in \mathbb{R}^n,$ 使得 $\lim\limits_{k\to\infty} x_k = x.$

由于对任意 $p \geqslant 1,$ 有 $x_{k+p} \in E_k,$ 那么 $x$ 是每个闭集 $E_k$ 的聚点, 从而 $x \in \overline{E}_k = E_k,$ 即 $x \in \bigcap\limits_{k=1}^{\infty} E_k.$

接下来证明 $x$ 的唯一性. 假设存在另一点 $y \in \bigcap\limits_{k=1}^{\infty} E_k,$ 则 $y \in E_k,$ 从而 $d(x, y) \leqslant \operatorname{diam} E_k$ 对所有 $k$ 成立, 从而 $d(x, y) = 0,$ 即 $x = y.$
\end{solution}

% \begin{question}[points = 15]
%   考虑定义在闭区间 $[0, 1]$ 上的函数
%   \[\varphi_0(x) = \begin{cases}
%     x, & \text{ 若 } ~ 0 \leqslant x \leqslant \dfrac{1}{2}, \\
%     1 - x, & \text{ 若 } ~ \dfrac{1}{2} < x \leqslant 1,
%   \end{cases}\]
%   并以 $1$ 为周期将其延拓为定义在整个实数域 $\mathbb{R}$ 上的函数 $\varphi(x).$ 设 $1 < q \leqslant p$ 为正整数. 定义函数 $f(x) = \sum\limits_{n=1}^{\infty} a^n\varphi(b^nx),$ 其中 $0 < a < 1,$ $b$ 是正整数.
%   \begin{enumerate}
%     \item 证明函数 $f(x)$ 是 $\mathbb{R}$ 上良定义的, 即喊数项级数 $\sum\limits_{n=1}^{\infty} a^n\varphi(b^nx)$ 的收敛域为 $\mathbb{R},$ 并证明 $f(x)$ 是 $\mathbb{R}$ 上的连续函数.
%     \item 进一步设 $ab \geqslant 1,$ 证明函数 $f(x)$ 在 $\mathbb{R}$ 上处处不可导.
%   \end{enumerate}

% % \noindent\scoringbox
% \end{question}

\begin{question}[points = 12]
设 $\sum\limits_{n=1}^{\infty} a_n$ 为数项级数, 令 $S_n = \sum\limits_{k=1}^{n} a_k$ 为其通项的前 $n$ 项和, $\sigma_n = \dfrac{1}{n} \sum\limits_{k=1}^{n} s_k$ 为数列 $\{ s_n \}$ 的前 $n$ 项均值.

\begin{enumerate}
\item 若数项级数 $\sum\limits_{n=1}^{\infty} a_n$ 收敛, 证明幂级数 $\sum\limits_{n=1}^{\infty} a_nx^n$ 的和函数在闭区间 $[0, 1]$ 上有定义 (即幂级数在此区间上收敛) 且连续.
\item 设 $\lim\limits_{n\to\infty} \sigma_n = A \in \mathbb{R}, A \neq 0.$ 证明幂级数 $\sum\limits_{n=1}^{\infty} n\sigma_n x^n,$ $\sum\limits_{n=1}^{\infty} s_n x^n,$ $\sum\limits_{n=1}^{\infty} a_n x^n$ 的收敛半径都大于等于 $1,$ 并证明等式 $\sum\limits_{n=1}^{\infty} a_n x^n = (1-x)^2 \sum\limits_{n=1}^{\infty} n\sigma_n x^n$ 在 $\lvert x \rvert < 1$ 时恒成立.
\item 利用 $(1-x)^{-1}$ 在 $\lvert x \rvert < 1$ 内的幂级数展开
\[(1-x)^{-1} = 1 + x + x^2 + \cdots\]
求函数 $\dfrac{1}{(1-x)^2}$ 的幂级数展开, 并验证等式 $1 = (1-x)^2 \sum\limits_{n=0}^\infty (n+1)x^n$ 在 $\lvert x \rvert < 1$ 时恒成立. 由此证明
\[\lim\limits_{x \to 1-} \sum\limits_{n=1}^{\infty} a_n x^n = A.\]
\end{enumerate}

% \noindent\scoringbox
\end{question}

\begin{solution}
\begin{enumerate}
\item 由题设知幂级数 $\sum\limits_{n=1}^{\infty} a_nx^n$ 在 $x = 1$ 处收敛, 于是由幂级数的 Abel 第二定理知幂级数 $\sum\limits_{n=1}^{\infty} a_nx^n$ 在任意闭区间 $[a, b] \subset* (-1, 1]$ 上一致收敛, 特别地在 $[0, 1]$ 上一致收敛, 从而和函数在 $[0, 1]$ 上连续.
\item 由于 $\lim\limits_{n\to\infty} \sigma_n = A \in \mathbb{R}, A \neq 0,$ 所以当 $n$ 充分大时有
\[\dfrac{\lvert A \rvert}{2} \leqslant \lvert \sigma_n \rvert \leqslant \dfrac{3\lvert A \rvert}{2},\]
由夹逼准则知 $\lim\limits_{n\to\infty} \nroot[n]{\lvert \sigma_n \rvert} = 1.$ 另一方面, 有 $\lim\limits_{n\to\infty} \nroot[n]{n} = 1,$ 于是
\[\varlimsup_{n\to\infty} \nroot[n]{\lvert n\sigma_n \rvert} = \lim\limits_{n\to\infty} \nroot[n]{\lvert \sigma_n \rvert} \cdot \lim\limits_{n\to\infty} \nroot[n]{n} = 1.\]
由 Cauchy-Hadamard 定理知, 幂级数 $\sum\limits_{n=1}^{\infty} n\sigma_n x^n$ 收敛半径等于 $1.$

由于 $\sum\limits_{n=1}^{\infty} n\sigma_n x^n$ 收敛半径等于 $1,$ 所以任取 $\lvert x \rvert < 1,$ $\sum\limits_{n=1}^{\infty} n\sigma_n x^n$ 与 $\sum\limits_{n=1}^{\infty} (n-1) \sigma_{n-1} x^n = x \cdot \sum\limits_{n=1}^{\infty} (n-1) \sigma_{n-1} x^{n-1}$ 都收敛, 其中约定 $\sigma_0 = 0.$ 于是
\[\sum\limits_{n=1}^{\infty} s_n x^n = \sum\limits_{n=1}^{\infty} n\sigma_n x^n - \sum\limits_{n=1}^{\infty} (n-1) \sigma_{n-1} x^n = (1-x) \sum\limits_{n=1}^{\infty} n\sigma_n x^n\]
也收敛. 于是幂级数 $\sum\limits_{n=1}^{\infty} s_n x^n$ 的收敛半径 $r$ 要满足 $r \geqslant \lvert x \rvert.$ 由于 $x$ 是任取的满足 $\lvert x \rvert < 1$ 的数, 因此有 $r \geqslant 1.$

用类似的方法可以算得, 当 $\lvert x \rvert < 1$ 时,
\[\sum\limits_{n=1}^{\infty} a_n x^n = (1 - x) \sum\limits_{n=1}^{\infty} s_n x^n\]
收敛, 进而推出 $\sum\limits_{n=1}^{\infty} a_n x^n$ 的收敛半径要大于等于 $1.$

我们将当 $\lvert x \rvert < 1$ 时证明成立的两式
\begin{gather*}
\sum\limits_{n=1}^{\infty} s_n x^n = (1-x) \sum\limits_{n=1}^{\infty} n\sigma_n x^n, \\
\sum\limits_{n=1}^{\infty} a_n x^n = (1 - x) \sum\limits_{n=1}^{\infty} s_n x^n.
\end{gather*}
综合起来, 便是 $\sum\limits_{n=1}^{\infty} a_n x^n = (1-x)^2 \sum\limits_{n=1}^{\infty} n\sigma_n x^n.$
\item 由 $(1-x)^{-1}$ 的幂级数展开
\[(1-x)^{-1} = 1 + x + x^2 + \cdots,\]
利用待定系数法
\begin{align*}
(1-x)^{-2} & = (1-x)^{-1} \cdot (1-x)^{-1} = (1 + x + x^2 + \cdots) \cdot (1 + x + x^2 + \cdots) \\
& = 1 + a_1x + a_2x^2 + \cdots,
\end{align*}
那么有 $a_nx^n = 1\cdot x^n + x\cdot x^{n-1} + \cdots + x^n\cdot 1 = (n+1) x^n,$
故有 $(1-x)^{-2} = \sum\limits_{n=0}^\infty (n+1) x^n,$ 并且容易看出其收敛半径大于等于 $(1-x)^{-1}$ 的收敛半径. 于是在 $\lvert x \rvert < 1$ 的范围内, 有
\[1 = (1-x)^2 \sum\limits_{n=0}^\infty (n+1) x^n.\]

下面证明 $\lim\limits_{x \to 1-} \sum\limits_{n=1}^{\infty} a_n x^n = A.$ 在 $\lvert x \rvert < 1$ 的范围内有
\begin{align*}
\sum\limits_{n=1}^{\infty} a_n x^n - A & = (1-x)^2 \sum\limits_{n=1}^{\infty} n\sigma_n x^n - A \\
& = (1-x)^2 \sum\limits_{n=1}^{\infty} n\sigma_n x^n - A (1-x)^2 \sum\limits_{n=0}^\infty (n+1) x^n \\
& = (1-x)^2 \sum\limits_{n=1}^\infty x^n (n \sigma_n - (n+1)A) - (1-x)^2 A \\
& = (1-x)^2 \sum\limits_{n=1}^\infty x^n n (\sigma_n - A) - (1-x)^2 \sum\limits_{n=1}^\infty x^n A - (1-x)^2 A \\
& = (1-x)^2 \sum\limits_{n=1}^\infty x^n n (\sigma_n - A) - (1 - x) A - (1-x)^2 A
\end{align*}
当 $x \to 1-$ 时, 上式后两项 $-(1 - x) A$ 与 $-(1-x)^2 A$ 都趋于 $0,$ 因此接下来只要证明 $\lim\limits_{x \to 1-} (1-x)^2 \sum\limits_{n=1}^\infty x^n n (\sigma_n - A) = 0$ 即可.

由于 $\lim\limits_{n\to\infty} \sigma_n = A,$ 所以 $\forall \varepsilon > 0,$ 存在正整数 $N,$ 使得对任意 $n > N,$ 有 $\lvert \sigma_n - A \rvert < \varepsilon / 2.$ 令
\[M = \max\limits_{1\leqslant n \leqslant N} n \lvert \sigma_n - A \rvert,\]
对于 $0 < x < 1,$ 有
\begin{align*}
\left\lvert (1-x)^2 \sum\limits_{n=1}^\infty x^n n (\sigma_n - A) \right\rvert & \leqslant (1-x)^2 \sum\limits_{n=1}^\infty x^n n \lvert \sigma_n - A \rvert \\
& = (1-x)^2 \sum\limits_{n=1}^N x^n n \lvert \sigma_n - A \rvert + (1-x)^2 \sum\limits_{n=N+1}^\infty x^n n \lvert \sigma_n - A \rvert \\
& \leqslant (1-x)^2 \sum\limits_{n=1}^N x^n M + (1-x)^2 \sum\limits_{n=N+1}^\infty x^n n \dfrac{\varepsilon}{2} \\
& \leqslant M (1-x)^2 \sum\limits_{n=1}^N x^n + \dfrac{\varepsilon}{2} (1-x)^2 \sum\limits_{n=0}^\infty x^n (n + 1) \\
& = M x (1 - x) (1 - x^N) + \dfrac{\varepsilon}{2} \leqslant M (1 - x) + \dfrac{\varepsilon}{2}
\end{align*}
对于取定的 $\varepsilon > 0,$ 进一步令 $1 - \dfrac{\varepsilon}{2M} < x < 1,$ 代入上式有
\[\left\lvert (1-x)^2 \sum\limits_{n=1}^\infty x^n n (\sigma_n - A) \right\rvert \leqslant M \cdot \dfrac{\varepsilon}{2M} + \dfrac{\varepsilon}{2} = \varepsilon.\]
这样就完成了 $\lim\limits_{x \to 1-} (1-x)^2 \sum\limits_{n=1}^\infty x^n n (\sigma_n - A) = 0$ 的证明.
\end{enumerate}
\end{solution}

\end{document}
