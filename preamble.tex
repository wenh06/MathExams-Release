% \usepackage{graphicx} % Required for inserting images
\usepackage{siunitx}
\usepackage{relsize}
\usepackage{ulem}
\usepackage{array}
\usepackage{float}
\usepackage{cancel}
\usepackage{stackengine}
\stackMath

\usepackage{pifont}
% \usepackage{fontsize}

\newCJKfontfamily\simfang{simfang.ttf}[Extension = .ttf, Path=fonts/]
\newCJKfontfamily\simhei{simhei.ttf}[Extension = .ttf, Path=fonts/]
\newCJKfontfamily\simkai{simkai.ttf}[Extension = .ttf, Path=fonts/]
\newCJKfontfamily\simsun{simsun.ttc}[Extension = .ttc, Path=fonts/]
\newCJKfontfamily\simli{SIMLI.TTF}[Extension = .TTF, Path=fonts/]

\setCJKmainfont[Path = fonts/, BoldFont = simhei.ttf, Mapping = full-stop]{simsun.ttc}
\setCJKsansfont[Path = fonts/, BoldFont = simhei.ttf, Mapping = full-stop]{simsun.ttc}

\newcommand{\chuhao}{\fontsize{42pt}{\baselineskip}\selectfont}
\newcommand{\xiaochuhao}{\fontsize{36pt}{\baselineskip}\selectfont}
\newcommand{\yihao}{\fontsize{26pt}{\baselineskip}\selectfont}
\newcommand{\erhao}{\fontsize{22pt}{\baselineskip}\selectfont}
\newcommand{\xiaoerhao}{\fontsize{18pt}{\baselineskip}\selectfont}
\newcommand{\sanhao}{\fontsize{16pt}{\baselineskip}\selectfont}
\newcommand{\sihao}{\fontsize{14pt}{\baselineskip}\selectfont}
\newcommand{\xiaosihao}{\fontsize{12pt}{\baselineskip}\selectfont}
\newcommand{\wuhao}{\fontsize{10.5pt}{\baselineskip}\selectfont}
\newcommand{\xiaowuhao}{\fontsize{9pt}{\baselineskip}\selectfont}
\newcommand{\liuhao}{\fontsize{7.5pt}{\baselineskip}\selectfont}
\newcommand{\xiaoliuhao}{\fontsize{6.5pt}{\baselineskip}\selectfont}
\newcommand{\qihao}{\fontsize{5.5pt}{\baselineskip}\selectfont}
\newcommand{\bahao}{\fontsize{5pt}{\baselineskip}\selectfont}

% \usepackage[fontsize=13pt]{fontsize}

\newcommand*{\nroot}[2][2]{\sqrt[\leftroot{-3}\uproot{3}#1]{#2}}

% \newcommand{\chemph}[1]{\CJKunderanysymbol[textformat=\bfseries, sep=0.1em]{0.2em}{\tiny$\triangle$}{#1}}

\newcommand{\chemph}[2][$\triangle$]{\CJKunderanysymbol[textformat=\bfseries, sep=0.1em]{0.2em}{\tiny#1}{#2}}

% https://tex.stackexchange.com/a/443445/163881
% \dunderline[<offset of line top>]{<thickness>}{<content>}
\newcommand\dunderline[3][-1pt]{{%
\sbox0{#3}%
\ooalign{\copy0\cr\rule[\dimexpr#1-#2\relax]{\wd0}{#2}}}}

% 行内公式统一按照行间的样式
\everymath{\displaystyle}

\allowdisplaybreaks
